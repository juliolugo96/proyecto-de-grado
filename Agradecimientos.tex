\begin{displayquote}``Las palabras nunca alcanzan cuando lo que hay que decir desborda el alma.'' (Julio Cortázar)
\end{displayquote}


Hoy, al redactar estas líneas, me hallo frente a frente a esta hoja incrédulo. Verdaderamente lo logré, luego de tantas dificultades y obstáculos, alcancé este objetivo. No sé qué pensar, actuar ni reaccionar, solo siento el éxtasis como un enervante filo de electricidad corre por mis arterias, y una mente que empieza a volar sutilmente hasta mi tierna niñez. Lo veo todo: me veo pequeño, no más de siete años, llorando amargamente porque mi madre se fue de viaje sin mí, y mis abuelos dándome bocados de helado de arequipe y secándome las lágrimas con dulzura, mientras me daban un consuelo que apretaba mi corazón y me empujaba a abrazarlos con más fuerza, y llorar solo para sentir más de ese precioso amor paternal. Creo que fue en esos instantes, ahora lejanos, donde reposa mi más pura felicidad y por donde debo comenzar a agradecer.
\par
A mis abuelos, doña Galanda González de Lugo y don Jesús Antonio Lugo, maestra normalista y cobrador, quienes se hicieron cargo de mí desde que era un niño y, sin ser su responsabilidad, se dedicaron a criarme con el mayor de los amores que un ser humano puede pedir.
\par
A mi madre, Galanda Norellys, y mis tíos Jesús Antonio y Claudio Antonio, a mi hermano Iya y mis primas-hermanas Aries y Elluz Angelina, por enseñarme tantas cosas y complementar la persona que terminé siendo el día de hoy, apoyarme en los momento de mayor necesidad e impulsarme a ser mejor compartiéndome todas sus experiencias.
\par
A la Universidad de Los Andes, porque me dio la oportunidad de conocer personas espléndidas, maestros insignes y me abrió las puertas al mundo real, como ninguna otra experiencia lo hubiese hecho antes. Destaco dentro de esto, porque así debe hacerse, a Gerard Páez Monzón, quien en mi momento de mayor necesidad confió en mí y me dio la oportunidad de crecer como nunca lo había hecho. A Rafael Rivas Estrada, porque sin él hubiese abandonado la carrera en el segundo semestre. A Alejandro Mujica, quien me hizo estar seguro de mí mismo, de que era bueno y debía confiar en mi talento. A José Luis Herrera Diestra, porque me exigió al límite como nunca ningún docente lo había hecho hasta entonces, y me hizo comprender el valor del esfuerzo y de la dedicación, que la disciplina siempre supera al talento.
\par
A mis hermanos y hermanas de vida, que me soportaron todos estos años y que fueron mis grandes maestros, cinceles que formaron lo que comenzó siendo un bloque de piedra adolescente, en bruto y desaliñado; y hoy es una obra que se erige con firmeza ante el mundo. Por encima de todos: Abraham, desde los 11 años estamos apoyándonos juntos en esto. Creo que no me alcanzan las palabras para expresar el cariño y la gratitud que sentimos mi familia y yo por ti.
\par
A mis compañeros de estudio, sin ustedes nada de esto hubiese sido posible. A los que tengo en mente en este momento, por orden cronológico: Andrés Enrique (Kike), mi pana, adaptarme a la facultad no hubiese sido posible sin ti. Jesús Alfonso, flaco, mi vida en este proceso no hubiese sido la misma sin esa buena vibra y tus consejos, eres una persona que se merece lo mejor en cada instante. Ana Isabel, eres una persona preciosa, me hiciste madurar emocionalmente como nadie antes, y empezar a ver el mundo con otros ojos. Darvich, por ser un hermano mayor para mí durante mucho tiempo, y mostrarme lo que es tener actitud ante la vida. Johnathan, maestro, puedo decir que antes de estudiar contigo, no sabía lanzar una línea de código decente, me hiciste creer que tener el mejor promedio era posible. Luis, por haber sido el estudiante más dotado y talentoso que alguna vez tuve como preparador, por ser uno de mis grandes amigos y confidentes. A Juan Diego, por ser siempre esa motivación para ser mejor y no detenerme nunca e incentivar el espíritu de competitividad que siempre he tenido dentro. Y finalmente, a quien debería tener su propio párrafo, Breytner, por simplemente ser mi mayor ejemplo de vida.
\par
No solo no hubiese sido nada sin ustedes, sino por toda la gente que estuvo desde el comienzo. Algunos, siguen hasta hoy, ¡gracias, totales!