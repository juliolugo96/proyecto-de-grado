\documentclass[12pt,oneside]{book}

\usepackage{ulamonog} %proyecto de grado
\usepackage[utf8]{inputenc}

\usepackage{longtable}


% Para texto en chino, japonés o coreano:
\usepackage{CJKutf8}

% Para imágenes, tablas más agradables visualmente y listas de chequeo en tablas:
\usepackage{booktabs}%
\usepackage{graphicx}%
\graphicspath{ {img/} }
\usepackage{float}
\usepackage{pifont}% http://ctan.org/pkg/pifont
\usepackage[flushleft]{threeparttable}
\usepackage{adjustbox}
\newcommand{\cmark}{\ding{51}}%
\newcommand{\xmark}{\ding{55}}%

% Para mostrar código fuente:
\usepackage{textcomp}
\usepackage{listings}
\lstset{
basicstyle=\small\ttfamily,
columns=flexible,
breaklines=true,
upquote=true,
showstringspaces=false
}

% Para mostrar cuadros de información/advertencias:
\usepackage{xcolor}
\usepackage[tikz]{bclogo}
\usepackage[framemethod=tikz]{mdframed}
\usepackage[many]{tcolorbox}

\definecolor{bgblue}{RGB}{245,243,253}
\definecolor{ttblue}{RGB}{91,194,224}
\definecolor{electricyellow}{rgb}{1.0, 1.0, 0.0}

\mdfdefinestyle{mystyle}{%
  rightline=true,
  innerleftmargin=10,
  innerrightmargin=10,
  outerlinewidth=3pt,
  topline=false,
  rightline=true,
  bottomline=false,
  skipabove=\topsep,
  skipbelow=\topsep
}

\newtcolorbox{blackcodebox}[1][]{
  breakable,
  title=#1,
  colback=white,
  colbacktitle=white,
  coltitle=black,
  fonttitle=\bfseries,
  bottomrule=0pt,
  toprule=0pt,
  leftrule=2pt,
  rightrule=2pt,
  titlerule=0pt,
  arc=0pt,
  outer arc=0pt,
  colframe=black,
}

\newtcolorbox{redwarningbox}[1][]{
  breakable,
  freelance,
  title=#1,
  colback=white,
  colbacktitle=white,
  coltitle=black,
  fonttitle=\bfseries,
  bottomrule=0pt,
  boxrule=0pt,
  colframe=white,
  overlay unbroken and first={
  \draw[red!75!black,line width=3pt]
    ([xshift=5pt]frame.north west) -- 
    (frame.north west) -- 
    (frame.south west);
  \draw[red!75!black,line width=3pt]
    ([xshift=-5pt]frame.north east) -- 
    (frame.north east) -- 
    (frame.south east);
  },
  overlay unbroken app={
  \draw[red!75!black,line width=3pt,line cap=rect]
    (frame.south west) -- 
    ([xshift=5pt]frame.south west);
  \draw[red!75!black,line width=3pt,line cap=rect]
    (frame.south east) -- 
    ([xshift=-5pt]frame.south east);
  },
  overlay middle and last={
  \draw[red!75!black,line width=3pt]
    (frame.north west) -- 
    (frame.south west);
  \draw[red!75!black,line width=3pt]
    (frame.north east) -- 
    (frame.south east);
  },
  overlay last app={
  \draw[red!75!black,line width=3pt,line cap=rect]
    (frame.south west) --
    ([xshift=5pt]frame.south west);
  \draw[red!75!black,line width=3pt,line cap=rect]
    (frame.south east) --
    ([xshift=-5pt]frame.south east);
  },
}

% Agrega apéndices:
\usepackage[titletoc]{appendix}

% Corrige problemas de fuentes:
\usepackage[T1]{fontenc}

\usepackage[activeacute,spanish]{babel}

% Colocar citas:
\usepackage{csquotes}

\usepackage{color}
\usepackage[colorlinks]{hyperref}

% ***************************************************************** %
% En el siguiente comando se pueden modificar:
% Titulo, Autor, Palabras clave
% ***************************************************************** %

% si se va a imprimir,
% se incluye al final despues de citecolor=blue, el comando draft=true
% por lo cual se descomenta la última línea y se comenta la anterior.
% Las siguientes son propiedades del pdf, más no cambian nada dentro del
% texto de la monografía
\hypersetup{pdftitle={Estudio e Implementación de una Plataforma de Software para el intercambio de valores entre startups},
pdfauthor={Julio Manuel Paredes Lugo},
pdfsubject={Proyecto de Grado}, % se deja igual, a menos que sea la propuesta
pdfkeywords={Startups, Intercambio de valores, Red Social, Plataforma, Desarrollo Web}, pdfstartview=FitH, citecolor=blue, draft=true}

% ***************************************************************** %
% FIN DE
% Titulo, Autor, Palabras clave
% ***************************************************************** %

% Si desea que no aparezca la lista de tablas o figuras descomente las siguientes lineas
% \nolistoftables%
% \nolistoffigures%
\usepackage[numbers, square]{natbib} %Agregado numbers para compatibilidad con la bibliografía estilo IEEEtranN

\sloppy

\begin{document}

\frontmatter

% ***************************************************************** %
% Portada y resumen
% ***************************************************************** %

% Si desea que el logo de la ULA aparezca en la parte superior,
% descomente la siguiente línea. Por defecto aparece en la parte inferior
\logoarriba{}

% Año en el cual se entrega el proyecto de grado o la propuesta
\copyrightyear{2022}

% Al final cuando hayan presentado, sin comentar,
% deberia ser el número de tesis presentada y la opción, IO por ejemplo
% (Actualmente, 11-03-08, no se está trabajando con esta metodología de llevar un
% número de proyecto para las tesis, por lo cual debe ir comentado)
%\numproy{23SC}

% En caso de hacer la propuesta, descomente la siguiente
% instrucción. Para el Proyecto de Grado debe comentarse. Modifica
% tanto la categoría de la monografía, como la aparición de
% "Presentado ante la ilustre Universidad de Los Andes
% como requisito parcial para obtener el Título de" en la portada
%\tipomonografia{Propuesta de Proyecto de Grado}

% Título de la monografía, el cual saldrá en la portada
\title{Estudio e Implementación de una Plataforma de Software para el intercambio de valores entre startups}

% Autor de la monografía, el cual saldrá en la portada
\author{Julio Manuel Paredes Lugo}

% La fechaentrega, presentaciondia, presentacionlugar
% y mencionespecial, es para aquel caso en el cual se
% vaya a utilizar la hoja del veredicto en el formato
% que se presenta aquí. El primerjurado, segundojurado,
% y cedula, son campos que pueden llenarse, pero sólo
% aparecerán si se utiliza la página del veredicto

% Cédula del autor de la monografía
\cedula{26.373.468}

% Tutor del Proyecto de Grado
\tutor{Prof.\ Dr.\ Gerard Páez Monzón}

% Cualquiera de los instructores que aparecen a continuación
% pueden comentarse o descomentarse según sea el caso:

%%% Cotutor del Proyecto de Grado
% \cotutor{Dra. propuesta}
%%% Asesor del Proyecto de Grado
% \asesor{Dr. Asesor}
%%% Asesor Industrial del Proyecto de Grado
% \asesorindustrial{Dr. Asesor Industrial}
%%% Tutor Industrial del Proyecto de Grado
% \tutorindustrial{Dr. Tutor Industrial}
%%% Jurados del Proyecto de Grado
\primerjurado{Prof.\ Junior Altamiranda}
\segundojurado{Prof.\ Rafael Rivas Estrada}

% ***************************************************************** %
% Si sabe la fecha de presentación
%
% con o sin comentar
% (Esta hoja es un formato para asentar la nota del proyecto de grado;
% sin embargo, la hoja que se utiliza para ese fin, se busca en la escuela
% días antes de la presentación)
% ***************************************************************** %
\fechaentrega{Enero 2022} % Si sabe cuando se presentó
\presentaciondia{15 de Enero de 2022} % si conoce exactamente el día
\presentacionlugar{Facultad de Ingeniería} % si conoce exactamente el lugar donde se presentó
% ***************************************************************** %
% FIN DE
% Si sabe la fecha de presentación
% ***************************************************************** %


% ***************************************************************** %
% Si tiene mencion especial
% con o sin comentar
% ***************************************************************** %
%\mencionespecial{Este proyecto fue seleccionado como \textbf{mejor
%proyecto de grado} de la Escuela de Ingeniería de Sistemas, en el
%IC aniversario de la Facultad de Ingeniería.} % si tiene mención especial
% (El texto puede cambiarse \mencionespecial{*********} según corresponda
% ***************************************************************** %
% FIN DE
% Si tiene mencion especial
% con o sin comentar
% ***************************************************************** %

% NO TOCAR si es Ingenieria de Sistemas
% \grado{Ingeniero Químico} % por defecto Ingeniero de Sistemas

%\signaturepage ----- NO TOCAR

%%%%%OJO%%%%% Arreglen esto según su opción
% Si es control y automatizacion se comentan las siguientes líneas. Si es de
% Sistemas Computacionales comenta la tercera. De Investigación de Operaciones
% comenta la segunda
%\opcion{Control y Automatización}
\opcion{Sistemas Computacionales}
%\opcion{Investigación de Operaciones}

% Aquí se escribe el resumen de la monografía
\resumen{Teniendo como objetivo la construcción de un sistema computacional que aloje y facilite el intercambio de valores entre emprendimientos dentro de la red para Ignis Gravitas, Inc., se ha propuesto la elaboración de una plataforma en línea que suministre a los internautas la capacidad de construir perfiles públicos para emprendimientos, con los cuales puedan interactuar e interrelacionarse con otros emprendedores y con profesionales de la industria, y sean capaces a su vez de realizar intercambios de valores de diversa índole con transacciones verificadas y seguras. En el presente trabajo se realiza una exposición completa sobre la metodología para diseñar y ejecutar el proceso de construcción de la plataforma antes descrita. Como enfoque metodológico para el desarrollo de este proyecto, se utilizó una metodología FDD (Feature Driven Development), enfocada al desarrollo de software. En lo que respecta a implementación, se hizo uso de Vue.js como marco de trabajo para la interfaz gráfica, desde la cual se mostraba la información extraída de una Base de Datos Relacional asociada al manejador PostgreSQL localizada en AWS (Amazon Web Services) mediante el servicio RDS (Relational Database Service), y de una localizada en un servicio independiente MongoAtlas. Dadas estas herramientas y su integración adecuada, se logró construir la plataforma en línea que cumple con los objetivos propuestos.
}

% Aquí se escriben las palabras claves de la monografía
\descriptores{Plataforma digital, Emprendimiento, Intercambio de valores, Ingeniería del Software, Red Social}

% Esto es para que salga la cota en la hoja del resumen. Actualmente,
% 11-03-08, en la Escuela de Ingeniería de Sistemas no es necesario
% buscar la cota previamente, sino que el Proyecto de Grado se entrega
% en la Escuela sin cota.
%\cota{IXD A01.1}

% Si desea eliminar la frase "Este trabajo fue procesado en LATEX"
% del resumen, descomente la siguiente línea
\sinlatex{}

% ***************************************************************** %
% FIN DE
% Portada y resumen
% ***************************************************************** %


% ***************************************************************** %
% Si tiene dedicatoria
% con o sin comentar
% ***************************************************************** %
\dedicatoria{A mi Abuela Galanda\\ y a mi Abuelo Jesús Antonio.}
% ***************************************************************** %
% FIN DE
% Si tiene dedicatoria
% ***************************************************************** %

\beforepreface %

% ***************************************************************** %
% Agradecimientos y capítulos NO numerados
% ***************************************************************** %

\prefacesection{Agradecimientos}
% Por cada \input{}, debe existir un archivo .tex con el nombre entre corchetes:
% Aún no hay agradecimientos, descomentar lo siguiente cuando los haya:
\begin{displayquote}``Somos como enanos a los hombros de gigantes. Podemos ver más, y más lejos que ellos, no porque la agudeza de nuestra vista ni por la altura de nuestro cuerpo, sino porque somos levantados por su gran altura.'' (Bernardo de Chartres)
\end{displayquote}

\begin{displayquote}
``Si he visto más lejos es porque estoy sentado sobre los hombros de gigantes.'' (Isaac Newton, entre otros)
\end{displayquote}

Este proyecto de grado simboliza la culminación de un largo y árduo camino tras el cual doy fin a una etapa de vida e inicio otra, oficialmente como Ingeniero de Sistemas, pero más allá de eso, como profesional en eterna formación y aprendizaje. No habría podido llegar acá sin la ayuda y guía de las siguientes personas (y de muchas otras en mis recuerdos), para las cuales nunca bastarán las palabras de agradecimiento que pueda tener.

\vspace{4mm}

Sin orden particular, aunque parezca:

\vspace{4mm}

A mi Padre Celestial. A mis santos, mis ``viejos'', mi ángel de la guarda. Gracias por sus bendiciones y cuidados, gracias por siempre mantenerme en el camino correcto y resguardarme de todo lo malo.
\par
A mis padres, porque, sin duda alguna (y muy literalmente) no estaría aquí hoy de no ser por ustedes. Por enseñarme que ``rendirse'' no es un verbo que puedo ni debo conjugar en primera persona, por los valores que inculcaron en mi y porque de ustedes nunca faltó el apoyo que necesitaba para levantarme tras los inevitables tropiezos que he dado al andar en mi camino. Soy lo que soy gracias a ustedes. Mis logros son suyos. A ustedes, siempre, mis eternas gracias y todo el amor que tengo para darles.
\par
A mis hermanos, porque de ustedes he descubierto que si lo imaginas, lo haces posible. Han sido mis guías en grandiosos paisajes y aunque les llevo ventaja en edad, he confiado mis pasos detrás de ustedes sin dudar por un instante. Gracias por ser. Gracias por estar.
\par
A toda mi familia, que siempre me ha apoyado y apoya en cada paso que doy, que son quienes me empujan a ser cada día mejor y a luchar por los sueños, porque me han enseñado y demostrado que ni siquiera el cielo es el límite y que si lo quiero, lo puedo.
\par
A la ilustre Universidad de Los Andes, de la cual orgullosamente formo parte y a mis profesores a lo largo de mi carrera profesional, porque independientemente de sus acciones u omisiones, han formado -cual cincel a bloque de piedra- al ingeniero que hoy ven ante ustedes. A todos los que tomaron en cuenta mi condición particular y tendieron una mano amiga o dieron invaluables consejos, e incluso a los que no, les doy mis más sinceras gracias. Menciones especiales a los profesores Demián Gutierrez, Oswaldo Ramirez, Fabiola Díaz, Luz Marina Pereira, y por último pero no por ello menos importante, a mi tutor, Rafael Rivas. Gracias por no permitirme desistir.
\par
A mi amada Sandra. Palabras faltan para decirte lo mucho que agradezco de corazón tu apoyo y ánimo cuando más lo necesitaba, porque me brindaste luz cuando sólo parecía encontrar oscuridad. Te amo. No habría podido llegar aquí sin ti.
\par
A mis amigos y compañeros de estudio, que hicieron de esta etapa universitaria un gran momento para recordar y que compartieron conmigo sus triunfos, derrotas y aprendizajes, mil gracias. A riesgo siempre de omitir (sin intención) algún nombre, pero no por ello apreciarles y quererles menos: Armando, Carla, Dario, Fernando, Gerardo, Idaí, Iramsaby, Jesús, Joel, Jorge, Junior, Karla, Liliberth, Luis, Manuel, Stefan, Syra, Totti, Vladimir.
\par
A todos, los que hoy comparten conmigo este logro, gracias totales.

%%% Capitulo sin numero, antes de la pagina 1

%\prefacesection{Introducción}
% Este es un ejemplo de una sección no numerada.


% ***************************************************************** %
% FIN DE
% Agradecimientos y capítulos NO numerados
% ***************************************************************** %

\afterpreface%

\pagestyle{fancyplain}
\renewcommand{\chaptermark}[1]{\markboth{#1}{\textsc{\footnotesize\thechapter\ #1}}}
\renewcommand{\sectionmark}[1]{\markright{\textsc{\footnotesize\thesection\ #1}}}
\lhead[\fancyplain{}{\textsc{\footnotesize\thepage}}]%
{\fancyplain{}{\rightmark}}
\rhead[\fancyplain{}{\leftmark}]%
{\fancyplain{}{\textsc{\footnotesize\thepage}}} \cfoot{}

\mainmatter%

% ***************************************************************** %
% Cuerpo
% ***************************************************************** %
% De aquí en adelante se desarrollan los capítulos numerados de la monografía

\chapter{Introducción}

En este capítulo se definen los antecedentes que son la base para la presentación del problema así como también el planteamiento de este último, la justificación del proyecto de grado, los objetivos y la metodología que encaminaron el desarrollo de la solución del mismo.

\section{Antecedentes}

El filósofo Immanuel Kant propuso a través de su teoría de la percepción, que nuestro conocimiento del mundo exterior depende de nuestras formas de percepción. Así como el cuerpo humano posee, en general, cinco sentidos universalmente conocidos que le ayudan a percibir el entorno que lo rodea, el estudio de los mismos lo ha llevado a investigar y desarrollar maneras de emular estos sentidos de forma artificial, con múltiples propósitos; entre ellos, el de proveer a entidades hechas por el hombre de la habilidad de reconocer el mundo a su alrededor, y en consecuencia, la capacidad de actuar en él.

Nuestra condición humana nos permite percibir la estructura en tres dimensiones del mundo a nuestro alrededor con aparente facilidad. Por ejemplo, con sólo ver alrededor en una habitación llena de cosas, usted podría contar e inclusive nombrar a cada uno de los objetos que le rodea; inclusive, podría adivinar la textura de los mismos sin necesidad de hacer uso del sentido del tacto. Así mismo, la percepción en tres dimensiones le permite juzgar con gran precisión la distancia desde su ubicación actual hasta cada objeto de interés, permitiéndole tocarlo, tomarlo o manipularlo si así lo desea. Esta percepción, que nosotros como seres humanos llamamos sentido de la vista, se efectúa a través de células especializadas que tienen receptores que reaccionan a estímulos específicos (en este caso, ondas de radiación electromagnética de longitudes específicas, que se registran como la sensación de la luz), ubicadas en nuestros ojos.

Si bien la descripción del sentido de la vista es -o parece ser- sencilla, se trata de un sentido sumamente complejo y de hecho, podría decirse que es uno de los sentidos más importantes para el ser humano, así como el más perfecto y evolucionado.

¿Por qué se habla de complejidad? Szeliski nos explica que, ``en parte, es porque la visión es un problema inverso, donde buscamos encontrar variables desconocidas dada información insuficiente para especificar totalmente la solución. Por tanto, debemos recurrir a modelos físicos y probabilísticos para discerner entre soluciones potenciales. Sin embargo, modelar el mundo visual en toda su complejidad es mucho más difícil que, por ejemplo, modelar el tracto vocal que produce sonidos hablados.'' \citep{RS:09}

Esta complejidad lo hace convertirse en un campo de estudio de gran importancia, cuya denominación, a los fines de la emulación mencionada anteriormente, es de la Visión Artificial, también conocida como Visión por Computador.

El inicio de la visión artificial, desde el punto de vista práctico, fue marcado por Larry Roberts, el cual, en 1961 creó un programa que podía ``ver'' una estructura de bloques, analizar su contenido y reproducirla desde otra perspectiva, demostrando así a los espectadores que esa información visual que había sido mandada al ordenador por una cámara, había sido procesada adecuadamente por él. \citep{bb68865}

\section{Definición del Problema}

El Laboratorio de Sistemas Discretos, Automatización e Integración, LaSDAI (ver más en el apéndice \ref{App:DescripcionLasdai}), en sus áreas de Visión por Computador y Robótica, desea estudiar alternativas de plataformas de software para poder utilizar robots autónomos, que provean soporte para sistemas de medición láser, infrarrojo o una combinación de ambos, mediante los cuales se pueda realizar medidas de distancias y así, poder generar mapas del entorno a través de dichas medidas.

Si bien se cuenta ya con dos plataformas robóticas (denominados ``LR1'' y ``LR2'') se carece de una plataforma programática común, con amplio soporte de la comunidad de investigación en robótica y de conocimiento en LaSDAI\@. Esta condición limita sustancialmente la investigación, el uso y la difusión de tecnologías afines a la robótica y la visión por computadora, dejando de lado este campo de investigación.

\section{Justificación}

LaSDAI posee y utiliza dos plataformas robóticas, los cuales cuentan cada uno con interfaces programáticas desarrolladas por separado. Esto, si bien es adecuado para el uso específico de cada plataforma, supone problemas de intercomunicación e interoperación, sin mencionar el costo en mantenimiento de dichas interfaces a nivel de código. Por ende, establecer una plataforma de software común para ambos, reduce a lo mínimo necesario la codificación personalizada para cada plataforma robótica, provee soporte al involucrar a un mayor número de personas y facilita el desarrollo de otras plataformas robóticas derivadas de esta.

\section{Objetivos}

\subsection{Objetivo General}

Investigar y desarrollar documentación adecuada que permita establecer una plataforma común de software para el manejo y navegación de robots móviles, que provea soporte a sensores tales como Microsoft Kinect, para obtener datos y realizar mediciones de entorno.

\subsection{Objetivos Específicos}

\begin{enumerate}
	\itemsep1pt \parskip1pt \parsep1pt
	\item Analizar las alternativas en plataformas de software disponibles para control robótico.
	\item Analizar el software disponible para elaboración de mapas de entorno.
	\item Analizar los requerimientos del módulo de creación de mapas.
	\item Generar un mapa de entorno mediante el software seleccionado.
	\item Realizar documentación de la estructura de los mapas generados mediante el software.
	\item Realizar documentación adecuada y actualizada para la difusión y posterior uso del software.
\end{enumerate}

\section{Metodología Utilizada}

Con la finalidad de llevar a cabo el desarrollo del proyecto de grado de forma eficiente y a la vez incorporar metodologías actuales enfocadas al desarrollo por parte de individuos (como es normalmente el caso en cuanto a proyectos de grado), se estudió el uso del método PSP~\citep{Humphrey200503} (Personal Software Process) mejorado con prácticas tomadas de los métodos de programación Ágiles, en particular, el método Extreme Programming, orientado a una sola persona, es decir, PXP~\citep{pxppaper} (Personal eXtreme Programming) integrado con el método Kanban.

Esto se llevó a cabo mediante las siguientes actividades realizadas:

\begin{itemize}
	\itemsep1pt \parskip1pt \parsep1pt
	\item Recolección de requerimientos.
	\item Planificación.
	\item Inicialización de iteración.
	\begin{itemize}
		\item Diseño.
		\item Implementación.
		\item Pruebas de sistema.
		\item Retrospectiva, análisis de resultados.
	\end{itemize}
	\item Finalización de iteración.
\end{itemize}

\chapter{Marco Teórico}

En este capítulo, se describen los fundamentos teóricos necesarios para el entendimiento y comprensión del proyecto. Se da una definición de los conceptos de robótica, robots, visión por computadora y la definición de las tareas realizadas por un robot autónomo para la navegación y ubicación de si mismo en un entorno. Se especifican las características técnicas del sistema Kinect y por último, se define la metodología Ágil para desarrollo de proyectos y los métodos PXP y Kanban para el desarrollo de aplicaciones, creando con esto una base teórica con la finalidad de tener una introducción del hardware y software utilizado, las herramientas de diseño y permitir al lector tener una idea de la naturaleza del contenido del resto del documento.

\section{Robótica}
La Robótica es aquella rama dentro de la Ingeniería que se ocupa de la aplicación de la informática al diseño y al uso de máquinas con el objetivo que de lo que de esto resulte pueda de alguna manera sustituir a las personas en la realización de determinadas funciones o tareas.
En palabras más simples, la robótica es la ciencia y la tecnología de los robots, porque básicamente se ocupa del diseño, manufactura y aplicaciones de los robots que crea. En la Robótica se combinan varias disciplinas al mismo tiempo, como la mecánica, la electrónica, la inteligencia artificial, la informática y la ingeniería de control, en tanto, también, por los quehaceres que desempeña, resulta fundamental el aporte que recibe y extrae de campos tales como el álgebra, los autómatas programables y las máquinas de estados.

\section{Robots}
El término robot alcanza su primera repercusión en la tercera década del siglo pasado, a instancias de R.U.R (\textit{Robots Universal Rossum}), una obra teatral de ciencia ficción escrita por el autor checo Karel Čapek, en la cual por primera vez se hace alusión al concepto de robot, extraído del término checo ``robota'', que significaba ``trabajos forzados''.

A su vez, el término ``robótica'' es acuñado por Isaac Asimov, definiendo a la ciencia que estudia a los robots. Asimov creó también las Tres Leyes de la Robótica, definidas de esta manera:

\begin{enumerate}
	\itemsep1pt \parskip1pt \parsep1pt
	\item Un robot no puede actuar contra un ser humano o, mediante la inacción, permitir que un ser humano sufra daños.
	\item Un robot debe obedecer las órdenes dadas por los seres humanos, salvo que estén en conflictos con la primera ley.
	\item Un robot debe proteger su propia existencia, a no ser que esté en conflicto con las dos primeras leyes.
\end{enumerate}

Desde sus comienzos como disciplina y como parte fundamental de la Ingeniería, la Robótica ha estado incansablemente buscando construir artefactos que materialicen el deseo humano de crear seres a su semejanza a quienes poder delegarles tareas, trabajos o actividades por demás pesadas y desagradables de llevar a cabo. Pero y aunque muchos ni se lo esperen, desde tiempos inmemoriales, muy, muy lejos de las computadoras, hubo unas cuantas expresiones de la robótica. Porque por ejemplo, los antiguos egipcios unieron brazos mecánicos a las estatuas de sus dioses y esgrimían que el movimiento de los miembros se llevaba a cabo por obra y gracias de estos, inclusive los griegos construyeron estatuas que operaban con sistemas hidráulicos, los cuales eran utilizados para fascinar a los adoradores de los templos.

Y también, aproximadamente entre los siglos XVII y XVIII, en Europa, se construyeron muñecos mecánicos muy ingeniosos que ostentaban algunas características como las que presentan los robots de la actualidad. En un constante e incansable ensayo a través de los siglos y cuando ya era un hecho la entrada en el nuevo milenio (2000), la empresa Honda Motor Co. Ltda. concretó a Asimo, el primer robot humanoide capaz de desplazarse de forma bípeda e interactuar con las personas.

La historia de la robótica va unida a la construcción de ``artefactos'', que trataban de materializar el deseo humano de crear seres a su semejanza y que lo descargasen del trabajo. El ingeniero español Leonardo Torres Quevedo (que construyó el primer mando a distancia para su automóvil mediante telegrafía sin hilo, el ajedrecista automático, el primer transbordador aéreo y otros muchos ingenios) acuñó el término ``automática'' en relación con la teoría de la automatización de tareas tradicionalmente asociadas.

\subsection{Clasificación de los robots}

En términos generales, un robot se clasifica por sus capacidades, así como también su área de operación, sus grados de autonomía o el fin con el que han sido construidos. Sin embargo, también pueden clasificarse en términos de la era en la que fue implementado, según su forma de construcción o la manera como son controlados. A continuación se exponen algunas formas comunes de clasificación:

\subsubsection{Según su cronología}
La que a continuación se presenta es la clasificación más común, respecto al área de robótica industrial:

\begin{description}
\item[1\textordfeminine\ Generación.]
Manipuladores. Son sistemas mecánicos multifuncionales con un sencillo sistema de control, bien manual, de secuencia fija o de secuencia variable.

\item[2\textordfeminine\ Generación.]
Robots de aprendizaje. Repiten una secuencia de movimientos que ha sido ejecutada previamente por un operador humano. El modo de hacerlo es a través de un dispositivo mecánico. El operador realiza los movimientos requeridos mientras el robot le sigue y los memoriza.

\item[3\textordfeminine\ Generación.]
Robots con control sensorizado. El controlador es una computadora que ejecuta las órdenes de un programa y las envía al manipulador para que realice los movimientos necesarios.

\item[4\textordfeminine\ Generación.]
Robots inteligentes. Son similares a los anteriores, pero además poseen sensores que envían información a la computadora de control sobre el estado del proceso. Esto permite una toma inteligente de decisiones y el control del proceso en tiempo real.
\end{description}

\subsubsection{Según su arquitectura}
La arquitectura, es definida por el tipo de configuración general del robot, puede ser metamórfica. El concepto de metamorfismo, de reciente aparición, se ha introducido para incrementar la flexibilidad funcional de un robot a través del cambio de su configuración por el propio robot.
El metamorfismo admite diversos niveles, desde los más elementales (cambio de herramienta o de efecto terminal), hasta los más complejos como el cambio o alteración de algunos de sus elementos o subsistemas estructurales.

Los dispositivos y mecanismos que pueden agruparse bajo la denominación genérica del robot, tal como se ha indicado, son muy diversos y es por tanto difícil establecer una clasificación coherente de los mismos que resista un análisis crítico y riguroso. La subdivisión de los robots, con base en su arquitectura, se hace en los siguientes grupos: poliarticulados, móviles, androides, zoomórficos e híbridos.

\begin{enumerate}
	\itemsep1pt \parskip1pt \parsep1pt
	\item Poliarticulados
	En este grupo se encuentran los robots de muy diversa forma y configuración, cuya característica común es la de ser básicamente sedentarios (aunque excepcionalmente pueden ser guiados para efectuar desplazamientos limitados) y estar estructurados para mover sus elementos terminales en un determinado espacio de trabajo según uno o más sistemas de coordenadas, y con un número limitado de grados de libertad. En este grupo, se encuentran los manipuladores, los robots industriales, los robots cartesianos y se emplean cuando es preciso abarcar una zona de trabajo relativamente amplia o alargada, actuar sobre objetos con un plano de simetría vertical o reducir el espacio ocupado en el suelo.

	\item Móviles
	Son robots basados en carros o plataformas, dotados de un sistema locomotor de tipo rodante y con gran capacidad de desplazamiento. Siguen su camino por telemando, guiándose por la información recibida de su entorno a través de sus sensores, a través de rutas o movimientos previamente planificadas y un híbrido entre estas dos últimas.

	\item Androides
	Son robots que intentan reproducir total o parcialmente la forma y el comportamiento cinemático del ser humano. Actualmente, los androides son todavía dispositivos muy poco evolucionados y sin utilidad práctica, y destinados, fundamentalmente, al estudio y experimentación. Uno de los aspectos más complejos de estos robots, y sobre el que se centra la mayoría de los trabajos, es el de la locomoción bípeda. En este caso, el principal problema es controlar dinámica y coordinadamente en el tiempo real el proceso y mantener simultáneamente el equilibrio del robot.

	\item Zoomórficos
	Los robots zoomórficos, que considerados en sentido no restrictivo podrían incluir también a los androides, constituyen una clase caracterizada principalmente por sus sistemas de locomoción que imitan a los diversos seres vivos. A pesar de la disparidad morfológica de sus posibles sistemas de locomoción es conveniente agrupar a los robots zoomórficos en dos categorías principales: caminadores y no caminadores. El grupo de los robots zoomórficos no caminadores está muy poco evolucionado. Los experimentos efectuados en Japón basados en segmentos cilíndricos biselados acoplados axialmente entre sí y dotados de un movimiento relativo de rotación. Los robots zoomórficos caminadores multípedos son muy numerosos y están siendo objeto de experimentos en diversos laboratorios con vistas al desarrollo posterior de verdaderos vehículos terrenos, piloteados o autónomos, capaces de evolucionar en superficies muy accidentadas. Las aplicaciones de estos robots serán interesantes en el campo de la exploración espacial y en el estudio de los volcanes.

	\item Híbridos
	Corresponden a aquellos de difícil clasificación, cuya estructura se sitúa en combinación con alguna de las anteriores ya expuestas, bien sea por conjunción o por yuxtaposición. Por ejemplo, un dispositivo segmentado articulado y con ruedas, es al mismo tiempo, uno de los atributos de los robots móviles y de los robots zoomórficos.
\end{enumerate}

\subsubsection{Según su modalidad de control}
La modalidad de control se refiere a la dependencia –o no– de un operador humano que instruya órdenes al robot; por tanto, se subdivide en dos grandes grupos:

\begin{enumerate}
	\itemsep1pt \parskip1pt \parsep1pt
	\item Teledirigidos
	Se define como robots teledirigidos a aquellos que necesitan la intervención de un operador humano, ya sea en forma parcial o total, por ejemplo los utilizados en la desactivación de explosivos. Es necesario destacar que algunos investigadores sugieren que el término robot no es adecuado cuando estos dispositivos son teledirigidos.
	\item Autónomos
	Se les llama autónomos a aquellos robots que son capaces de tomar sus propias decisiones basados en la comprensión del entorno en que se encuentren. Existen numerosos tipos de robots de variadas configuraciones que se encuadran en esta categoría, como es el caso de un brazo robot con visión artificial sin asistencia humana realizando tareas de clasificación de objetos, o de los robots móviles del tipo vehículo.
\end{enumerate}

\subsection{Robots Autónomos}
Un robot autónomo es un robot que realiza comportamientos o tareas con un alto grado de autonomía, que es particularmente deseable en campos tales como la exploración del espacio, la limpieza de suelos, cortar el césped, y el tratamiento de aguas residuales.

Algunos robots de fábricas modernas son ``autónomos'' dentro de los límites estrictos de su entorno directo. Puede que no sea la existencia de todos los grados de libertad en su entorno, pero el lugar de trabajo del robot de la fábrica es un reto y, a menudo puede contener, variables caóticas e impredecibles. La orientación exacta y la posición del siguiente objeto de trabajo e incluso (en las fábricas más avanzadas) el tipo de objeto y la tarea requerida debe ser determinado. Esto puede variar de manera impredecible (por lo menos desde el punto de vista del robot).

Un área importante de la investigación robótica es permitir que el robot pueda hacer frente a su entorno ya sea en tierra, bajo el agua, en el aire, bajo tierra o en el espacio.

Un robot completamente autónomo puede:

\begin{itemize}
	\itemsep1pt \parskip1pt \parsep1pt
	\item Obtener información sobre el medio ambiente (Regla \#1)
	\item Trabajar por un período prolongado sin intervención humana (Regla \#2)
	\item Mover todo o parte de sí mismo a través de su entorno operativo sin ayuda humana (Regla \#3)
	\item Evitar situaciones que son perjudiciales para las personas, los bienes, o sí mismo, si esos son parte de sus especificaciones de diseño (Regla \#4)
\end{itemize}

Un robot autónomo también puede aprender o adquirir nuevos conocimientos como ajustarse a nuevos métodos para llevar a cabo sus tareas o adaptarse a un entorno cambiante.

Al igual que otras máquinas, los robots autónomos todavía requieren de un mantenimiento regular.

Ejemplos:

\subsubsection{Automantenimiento}
El primer requisito para la autonomía física completa es la capacidad de un robot para cuidar de sí mismo. Muchos de los robots que funcionan con baterías en el mercado hoy en día pueden encontrar y conectarse a una estación de carga, y algunos juguetes como Aibo de Sony son capaces de realizar auto-acoplamiento para cargar sus baterías.

El mantenimiento realizado se basa en la ``propiocepción'', o la capacidad de sentir el propio estado interno. En el ejemplo de carga de la batería, el robot puede decir propioceptivamente que sus baterías están bajas y en consecuencia, buscar el cargador. Otro sensor propioceptivo es común para la supervisión de calor. El aumento de la propiocepción se requerirá para los robots para trabajar de forma autónoma, cerca de la gente y en ambientes hostiles. Las propiocepciones comunes incluyen sensores de detección térmica, óptica y háptica, así como el efecto Hall (eléctrica).

\subsubsection{Sintiendo el medio ambiente}
La exterocepción es la detección de información del medio ambiente. Los robots autónomos deben tener una gama de sensores ambientales para llevar a cabo su tarea y no meterse en problemas.

Los sensores exteroceptivos comunes incluyen el espectro electromagnético, el sonido, el tacto, química (olor), la temperatura, la distancia hacia múltiples objetos, y la altitud.
Algunas cortadoras de césped robóticas adaptarán su programación mediante la detección de la velocidad en la que la hierba crece a medida que sea necesario para mantener un césped perfectamente cortado, y algunos robots de limpieza por aspiración tienen detectores de tierra que detectan la cantidad de suciedad que se está recogido y utilizan esta información para decirles que deben permanecer en un área durante más tiempo.

\subsubsection{Desempeño de tareas}
El siguiente paso en el comportamiento autónomo es el de llevar realmente a cabo una tarea física. Una nueva área que muestra promesa comercial es la de los robots domésticos, con una avalancha de pequeños robots aspiradora que comienzan con iRobot y Electrolux en 2002. Si bien el nivel de inteligencia no es muy alta en estos sistemas, ellos pueden navegar en zonas extensas y conducirse a si mismos en situaciones estrechas alrededor de los hogares utilizando sensores de contacto y sin contacto. Ambos robots utilizan algoritmos propietarios para aumentar la cobertura por encima del simple rebote al azar.

El siguiente nivel de ejecución de la tarea autónoma requiere que un robot pueda realizar tareas condicionales. Por ejemplo, los robots de seguridad se pueden programar para detectar intrusos y responder de una manera particular, dependiendo de donde esté ubicado el intruso.

\subsubsection{Navegación interior}
Para que un robot pueda asociar comportamientos con un lugar (localización) requiere saber dónde está y ser capaz de navegar de punto a punto. Tal navegación comenzó mediante guía cableada en la década de 1970 y progresó en la década de 2000 a la triangulación mediante balizas. Los robots autónomos comerciales actuales navegan basándose en la detección de características naturales.

Los primeros robots comerciales en lograrlo fueron el robot de hospital HelpMate de Pyxus y el robot guardia CyberMotion, ambos diseñados por pioneros en robótica en la década de 1980. Estos robots utilizaban originalmente planos de piso CAD creados manualmente, detección mediante sonar y variaciones de seguimiento de paredes para navegar a través de edificios. La próxima generación, tales como PatrolBot y la silla de ruadas autónoma de MobileRobots, ambos introducidos en 2004, tienen la capacidad de crear sus propios mapas basados en láser de un edificio y navegar por zonas abiertas, así como corredores. Su sistema de control cambia su ruta sobre la marcha si algo bloquea el camino.

Inicialmente, la navegación autónoma se basaba en sensores de geometría plana, como los telémetros láser, que sólo pueden realizar detecciones en un plano o nivel. Los sistemas más avanzados ahora fusionan la información de diversos sensores, tanto para la localización (posición) y la navegación. Los sistemas tales como Motivity pueden contar con diferentes sensores en diferentes áreas, dependiendo de lo que proporcione los datos más fiables en el momento, y pueden volver a generar mapas de entorno de forma autónoma.

En lugar de subir escaleras, lo que requiere hardware altamente especializado, la mayoría de los robots de interior navegan en áreas accesibles para minusválidos, controlando ascensores y puertas electrónicas. Con este tipo de interfaces de control de acceso, los robots ya pueden navegar libremente en el interior. Subir o bajar escaleras de forma autónoma y abrir puertas de forma manual, son temas actuales de investigación.

A medida que estas técnicas en interiores se siguen desarrollando, los robots aspiradora adquirirán la capacidad de limpiar una habitación específica designada por el usuario o una planta entera. Los robots de seguridad podrán rodear intrusos de forma cooperativa e incluso cortarles las salidas. Estos avances también traen protecciones asociadas: los mapas internos de los robots permiten típicamente la definición de ``zonas prohibidas'' para evitar que los mismos entren de forma autónoma en ciertas regiones.

\paragraph{Navegación en exteriores}
La autonomía al aire libre se logra más fácilmente en el aire, ya que los obstáculos son raros. Los misiles de crucero son robots altamente autónomas y bastante peligrosos. Los aviones no tripulados se utilizan cada vez más para el reconocimiento. Algunos de estos vehículos aéreos no tripulados (UAV) son capaces de volar toda su misión sin ninguna interacción humana en absoluto, excepto posiblemente para el aterrizaje cuando una persona interviene mediante control remoto por radio. Sin embargo, algunos aviones son capaces de realizar aterrizajes automáticos y seguros.

La autonomía al aire libre es más difícil para los vehículos de tierra, debido a:

\begin{itemize}
	\itemsep1pt \parskip1pt \parsep1pt
	\item La tridimensionalidad del terreno,
	\item Grandes disparidades en la densidad de superficie
	\item Exigencias climáticas
	\item Inestabilidad del medio ambiente detectado
\end{itemize}

En Estados Unidos, el proyecto MDARS (\textit{Mobile Detection Assessment and Response System}), que definió y construyó un robot prototipo de vigilancia exterior en la década de 1990, se está llevando a producción y será implementado en 2006. El robot MDARS de General Dynamics puede navegar de forma semi-autónoma y detectar intrusos, utilizando la arquitectura de software MRHA (\textit{Multiple Robot Host Architecture}) planeada para todos los vehículos militares no tripulados. El robot Seekur fue el primer robot comercial para demostrar las capacidades similares a MDARS para uso general por los aeropuertos, plantas de servicios públicos, instalaciones correccionales y Seguridad Nacional.

Los rovers MER-A y MER-B (conocidos actualmente como los rovers Spirit y Opportunity) pueden encontrar la posición del sol y navegar sus propias rutas a destinos sobre la marcha a través de:

\begin{itemize}
	\itemsep1pt \parskip1pt \parsep1pt
	\item Cartografía de la superficie mediante visión en 3D
	\item Cálculo de zonas seguras e inseguras en la superficie dentro de ese campo de visión
	\item Cálculo de rutas óptimas en toda la zona segura hacia el destino deseado
	\item Conducción a lo largo de la ruta calculada
	\item La repetición de este ciclo hasta que el destino se alcanza, o no haya ninguna ruta conocida hacia el destino
\end{itemize}

El Rover de ESA en planificación, ExoMars Rover, es capaz de localización relativa basada en visión y localización absoluta para navegar de forma autónoma a trayectorias seguras y eficaces a objetivos a través de:

\begin{itemize}
	\itemsep1pt \parskip1pt \parsep1pt
	\item La reconstrucción de modelos 3D del terreno que rodea al Rover con un par de cámaras estéreo
	\item La determinación de las zonas seguras e inseguras del terreno y la ``dificultad'' general para el Rover para navegar por el terreno
	\item Cálculo de caminos eficientes a través de la zona de seguridad hacia el destino deseado
	\item Conducir el Rover a lo largo del camino planeado
	\item La creación de un mapa de navegación de todos los datos de navegación anterior
\end{itemize}

El DARPA Grand Challenge y DARPA Urban Challenge han alentado el desarrollo de capacidades aún más autónomas para vehículos de tierra, mientras que este ha sido el objetivo demostrado por robots aéreos desde 1990 como parte de la AUVSI Internacional Aerial Robotics Competition.

\section{SLAM}
\label{sec:slam}
Para un robot autónomo, capaz de navegar por si mismo en un entorno, una de las tareas más desafiantes que puede realizar es la de estimar su posición y orientación en el ambiente en el cual navega, especialmente si no se cuenta con información previa sobre ese ambiente. De esto se trata SLAM, acrónimo en inglés para \textit{Simultaneous Localization And Mapping}, o Localización y Mapeo Simultáneos.

\subsection{Introducción}
El problema de construcción de mapas y localización simultáneos pregunta si es posible que un vehículo autónomo comience en una ubicación desconocida dentro de un entorno desconocido y que construya de forma incremental un mapa de este entorno mientras que usa este mapa de forma simultánea para calcular de forma absoluta la localización o ubicación del vehículo.

La solución para este problema es, en múltiples aspectos, el ``Santo Grial'' de la comunidad de investigación de vehículos autónomos. Por ello, la principal ventaja de SLAM es que elimina la necesidad de infraestructuras artificiales o de poseer conocimiento topográfico a priori del entorno.

\subsection{Historia}
El problema general de SLAM ha sido objeto de considerable investigación desde el inicio de una comunidad de investigación de la robótica y de hecho antes de este en áreas como sistemas de navegación de vehículos tripulados y estudios geofísicos. Se han propuesto varios enfoques para abordar tanto el problema de SLAM y también problemas de navegación más simplificados donde se hacen disponible informaciones adicionales de mapa o de ubicación del vehículo.

En términos generales, estos enfoques adoptan una de tres filosofías principales. La más popular de ellas es la aproximación estimación-teoría o basada en el filtro de Kalman. La popularidad de este enfoque se debe a dos factores principales. En primer lugar, proporciona directamente tanto una solución recursiva para el problema de navegación y de una manera de computar estimaciones consistentes para la incertidumbre en ubicaciones de vehículos y referencias de mapas sobre la base de modelos estadísticos para el movimiento del vehículo y observaciones relativas de referencias. En segundo lugar, un corpus sustancial del método y la experiencia ha sido desarrollado en el sector aeroespacial, marítimo y otras aplicaciones de navegación, de las que la comunidad autónoma de vehículos puede extraer.

Una segunda filosofía es la de evitar la necesidad de estimaciones de posición absoluta y de medidas precisas de la incertidumbre y en lugar de ello, emplear conocimiento más cualitativo de la ubicación relativa de las referencias y el vehículo para construir mapas y guiar el movimiento. El enfoque cualitativo para la navegación y el problema general de SLAM tiene muchas ventajas potenciales sobre la metodología de estimación-teoría en términos de limitar la necesidad de modelos precisos y los requisitos computacionales resultantes, y en su significativo ``atractivo antropomórfico''.

La tercera y muy amplia filosofía, elimina el filtro de Kalman o el riguroso formalismo estadístico al tiempo que conserva un enfoque esencialmente numérico o computacional para el problema de navegación y SLAM. Tales enfoques incluyen el uso de pareos de lugares de interés icónicos, el registro de un mapa global, regiones delimitadas y otras medidas para describir la incertidumbre. Un trabajo notable en esta filosofía se ha realizado mediante el uso de un enfoque bayesiano para mapeo de edificios que no asume las distribuciones de probabilidad de Gauss como es requerido por el filtro de Kalman. Esta técnica, aunque muy eficaz para la localización con respecto a los mapas, no se presta para proporcionar una solución gradual a SLAM, donde un mapa se construye gradualmente a medida que se recibe información de los sensores.
\cite{dissanayake01solution}

\section{Visión por Computadora}
Es el campo de la Inteligencia Artificial enfocado a que las computadoras puedan extraer información a partir de imágenes, ofreciendo soluciones a problemas del mundo real. La visión para los humanos no es ningún problema, pero para las máquinas es un campo muy complicado. Influyen texturas, luminosidad, sombras, objetos complejos, etc.

El propósito de la visión artificial o por computadora, es programar un computador para que ``entienda'' una escena o las características de una imagen.

Los objetivos típicos de la visión artificial incluyen:

\begin{itemize}
	\itemsep1pt \parskip1pt \parsep1pt
	\item La detección, segmentación, localización y reconocimiento de ciertos objetos en imágenes (por ejemplo, caras humanas).
	\item La evaluación de los resultados (por ejemplo, segmentación, registro).
	\item Registro de diferentes imágenes de una misma escena u objeto, es decir, hacer concordar un mismo objeto en diversas imágenes.
	\item Seguimiento de un objeto en una secuencia de imágenes.
	\item Mapeo de una escena para generar un modelo tridimensional de la escena; este modelo podría ser usado por un robot para navegar por la escena.
	\item Estimación de las posturas tridimensionales de humanos.
	\item Búsqueda de imágenes digitales por su contenido.
\end{itemize}

Estos objetivos se consiguen por medio de reconocimiento de patrones, aprendizaje estadístico, geometría de proyección, procesamiento de imágenes, teoría de grafos y otros campos. La visión artificial cognitiva está muy relacionada con la psicología cognitiva y la computación biológica.

\section{Microsoft Kinect}
Kinect para Xbox 360, o simplemente Kinect (originalmente conocido por el nombre en clave ``Project Natal''), es un controlador de juego libre y entretenimiento creado por Alex Kipman, desarrollado por Microsoft para las videoconsolas Xbox 360 y Xbox One, y desde junio del 2011 para PC a través de Windows 7 y Windows 8.

Kinect permite a los usuarios controlar e interactuar con la consola sin necesidad de tener contacto físico con un controlador de videojuegos tradicional, mediante una interfaz natural de usuario que reconoce gestos, comandos de voz, y objetos e imágenes. El dispositivo tiene como objetivo primordial aumentar el uso de la Xbox 360, más allá de la base de jugadores que posee en la actualidad.

En sí, Kinect compite con los sistemas Wiimote con Wii MotionPlus y PlayStation Move, que también controlan el movimiento para las consolas Wii y PlayStation 3, respectivamente.

El nombre en clave ``Proyecto Natal'' responde a la tradición de Microsoft de utilizar ciudades como nombres en clave. Alex Kipman, director de Microsoft, quien incubó el proyecto, decidió ponerle el nombre de la ciudad brasileña Natal como un homenaje a su país de origen y porque la palabra natal significa ``de o en relación al nacimiento'', lo que refleja la opinión de Microsoft en el proyecto como ``el nacimiento de la próxima generación de entretenimiento en el hogar''. Poco antes de la E3 2010 varios \textit{weblogs} tropezaron con un anuncio que supuestamente se filtró en el sitio italiano de Microsoft de que sugirió el título ``Kinect'', que confirmó más tarde.

\begin{figure}[h]
\centering
\includegraphics[width=0.75\textwidth]{kinectxbox360}
\caption{Microsoft XBOX 360 Kinect}
\end{figure}

\subsection{Características}
El sensor de Kinect es una barra horizontal de aproximadamente 23 cm (9 pulgadas) conectada a una pequeña base circular con un eje de articulación de rótula, y está diseñado para ser colocado longitudinalmente por encima o por debajo de la pantalla de vídeo.

El dispositivo cuenta con una cámara RGB, un sensor de profundidad, un micrófono de múltiples matrices y un procesador personalizado que ejecuta el software patentado, que proporciona captura de movimiento de todo el cuerpo en 3D, reconocimiento facial y capacidades de reconocimiento de voz.

\begin{figure}[h]
\centering
\includegraphics[width=0.75\textwidth]{kinectcomponents}
\caption{Componentes del Kinect}
\end{figure}

El sensor contiene un mecanismo de inclinación motorizado y en caso de usar una PC o un modelo original de Xbox 360, tiene que ser conectado a una toma de corriente a través de un adaptador, ya que la corriente que puede proveerle el cable USB es insuficiente; para el caso del modelo de Xbox 360 S esto no es necesario ya que esta consola cuenta con una toma especialmente diseñada para conectar el Kinect y esto permite proporcionar la corriente necesaria que requiere el dispositivo para funcionar correctamente.

El sensor de profundidad es un proyector de infrarrojos combinado con un sensor CMOS monocromo que permite a Kinect ver la habitación en 3D en cualquier condición de luz ambiental. El rango de detección de la profundidad del sensor es ajustable gracias al software de Kinect capaz de calibrar automáticamente el sensor.

\subsection{Especificaciones Técnicas}
Sus especificaciones \cite{openkinecthardware} \cite{ifixitkinect} más relevantes son:

\begin{itemize}
	\itemsep1pt \parskip1pt \parsep1pt
	\item Cámara / sensor infrarrojo: Microsoft / X853750001 / VCA379C7130 / MT9M001
	\item Rango efectivo (aproximado): 1,2 m – 3,5 m (puede ser menor o mayor, dependiendo de las condiciones ambientales)
	\item Resolución: 640x480 píxeles @ 30 Hz (profundidad de 11 bits: 2048 niveles de sensibilidad)
	\item Cámara RGB: VNA38209015 / MT9M112 / MT9v112
	\item Resolución: 640x480 píxeles @ 30 Hz (VGA de 8 bits)
	\item Emisor / proyector infrarrojo: OG12 / 0956 / D306 / JG05A
	\item Proyector láser de 830 nm.
	\item Potencia de salida: 60\~ mW
\end{itemize}

\subsection{Requerimientos para uso en Robótica}

El Kinect requiere 12 voltios \cite{openkinecthardware} para su funcionamiento (8 V como mínimo); es por ello que, tal como se mencionó anteriormente, la versión para PC cuenta con un adaptador de voltaje adicional, ya que el voltaje proporcionado por la conexión USB es por especificación de 5 voltios.

Es posible encontrar diversos tutoriales en línea \cite{batterypoweredkinect} sobre cómo alimentar al Kinect con la energía necesaria para manejarlo desde un robot autónomo y eliminar la necesidad de utilizar una extensión, asistiendo en su transportabilidad.

\section{Mapas de Entorno}

Uno de los objetivos de un robot autónomo es el de ubicarse a si mismo en el entorno para poder tener la capacidad de interactuar con él, a través de la elaboración de un mapa del mismo. Un mapa es la representación gráfica y métrica, por lo general en un plano, de un territorio o edificio, donde se representan las características físicas del mismo en base a una escala común. En nuestro caso particular, se denominará entonces mapa de entorno a la representación gráfica digitalizada de un entorno perteneciente al interior de un edificio, ya sea un salón, una oficina, un pasillo o demás similares.

\subsection{Tipos de Mapas de Entorno}

La clasificación de un mapa está altamente relacionada con el tipo de uso que pretenda dársele al mismo. \cite{Lee200309} Por ello, podemos clasificarles en una jerarquía según la fortaleza del mapa. En este contexto, la fortaleza del mapa proviene del rango de propiedades geométricas que pueden ser derivadas del mapa; éstas son:

\begin{itemize}
	\itemsep1pt \parskip1pt \parsep1pt
	\item Ubicaciones reconocibles: Consiste en una lista de ubicaciones que pueden ser reconocidas de forma confiable por el robot. No se pueden recuperar relaciones geométricas.
	\item Mapa topológico: Además de contener ubicaciones reconocibles, el mapa guarda cuáles ubicaciones están conectadas por caminos o rutas transitables. Se pueden recuperar las rutas entre ubicaciones ya visitadas.
	\item Mapa topológico métrico: Este término es usado para mapas en el que la información de distancias y ángulos es almacenada junto a las descripciones de las rutas. Se puede recuperar información métrica de las rutas que se han recorrido.
	\item Mapa métrico completo: La ubicación de los objetos se especifica en un sistema de coordenadas fijo. Se puede recuperar información métrica de cualquier objeto en el mapa.
\end{itemize}

Para el propósito de este proyecto, es necesario el uso de un mapa métrico completo, debido a que el robot debe ser capaz de moverse entre espacios libres, evitando obstáculos, teniendo como limitante la ausencia de características distintivas en una ubicación particular (por habitaciones o entornos similares, entornos variables por movimiento de objetos o personas, etc.) y pudiendo contar solamente con las relaciones métricas entre los objetos en el entorno. Podrían utilizarse puntos de referencia, pero sólo como objetos cuya ubicación fuese conocida y que pudiesen ser detectados remotamente (quizás para realizar triangulación).

\subsection{Mapa de Cuadrícula de Ocupación}

Ahora bien, ya se ha mencionado que el mapa debe ser generado a través del propio robot para que éste pueda ser autónomo. Esto introduce una serie de dificultades relacionadas con la exactitud de las mediciones obtenidas por los sensores, tal como se hizo referencia en la sección~\ref{sec:slam}. Una forma de mitigar dichas dificultades tiene que ver con la generación de mapas de cuadrículas de ocupación, que se refiere a una familia de algoritmos probabilísticos cuya idea básica es la de representar un mapa del entorno basado en variables binarias separadas de forma uniforme, donde cada una representa la presencia o ausencia de un obstáculo en esa ubicación del entorno.

\section{Desarrollo Ágil de Proyectos}

En todo proyecto, el objetivo deseado es (o debería ser) el de maximizar la eficiencia y disminuir el costo, ya sea monetario o de tiempo. Para lograr este objetivo se han propuesto diferentes formas de administrar los proyectos, con infinidad de niveles de control, desde prácticamente ningún control en lo absoluto, hasta la microgerencia de las tareas más mínimas posibles.

El desarrollo de un trabajo especial de grado, tesis, o proyecto de grado puede verse, tal como se denominó por último, como un proyecto en si mismo, donde el objetivo es cumplir con los objetivos generales y específicos dentro de un marco de tiempo determinado, por lo que puede aplicarse lo mencionado al inicio para cumplir con los objetivos de la mejor manera posible.

Así pues, en el área de la ingeniería de sistemas y el desarrollo de software, se han empleado igualmente innumerables métodos para llevar a cabo la gerencia de un proyecto; sin que esto pretenda convertirse en un estudio a profundidad de dichos métodos, se nombra a continuación uno de los más populares en la actualidad, junto algunos métodos que emplean su filosofía y cuyo fin no es otro que el de organizar el desarrollo del proyecto y así poder rendir cuentas del mismo.

Las metodologías Ágiles de desarrollo, fueron propuestas en el año 2001 por diversos representantes de múltiples metodologías de desarrollo, tales como SCRUM, Programación Extrema, DSOM, Desarrollo de Software Adaptativo, Crystal y otros, con el fin de encontrar un terreno común que involucrara una alternativa al desarrollo de software pesado y orientado a la documentación.

Su idea consistió en enfocar el proceso de desarrollo en las personas en vez de al proceso en sí, reduciendo la burocracia y los procesos de planificación, documentación y modelado al mínimo posible, a través de principios documentados en un ``manifiesto''.

\subsection{Principios del Desarrollo Ágil}

El manifiesto para el Desarrollo Ágil de Software, indica el enfoque a seguir por esta metodología. Sus valores principales son los siguientes:

\begin{itemize}
	\itemsep1pt \parskip1pt \parsep1pt
	\item Individuos e interacciones sobre procesos y herramientas
	\item Software funcionando sobre documentación extensiva
	\item Colaboración con el cliente sobre negociación contractual
	\item Respuesta ante el cambio sobre seguir un plan
\end{itemize}

En las propias palabras del manifiesto: ``Esto es, aunque valoramos los elementos de la derecha, valoramos más los de la izquierda''.\cite{beck2001manifiesto}

Los principios producto de estos valores son:\cite{beck2001principios}

\begin{itemize}
	\itemsep1pt \parskip1pt \parsep1pt
	\item Nuestra mayor prioridad es satisfacer al cliente mediante la entrega temprana y continua de software con valor.
	\item Aceptamos que los requisitos cambien, incluso en etapas tardías del desarrollo. Los procesos Ágiles aprovechan el cambio para proporcionar ventaja competitiva al	cliente.
	\item Entregamos software funcional frecuentemente, entre dos semanas y dos meses, con preferencia al periodo de tiempo más corto posible.
	\item Los responsables de negocio y los desarrolladores	trabajamos juntos de forma cotidiana durante todo el proyecto.
	\item Los proyectos se desarrollan en torno a individuos motivados. Hay que darles el entorno y el apoyo que necesitan, y confiarles la ejecución del trabajo.
	\item El método más eficiente y efectivo de comunicar información al equipo de desarrollo y entre sus miembros es la conversación cara a cara.
	\item El software funcionando es la medida principal de	progreso.
	\item Los procesos Ágiles promueven el desarrollo sostenible. Los promotores, desarrolladores y usuarios
	debemos ser capaces de mantener un ritmo constante de forma indefinida.
	\item La atención continua a la excelencia técnica y al	buen diseño mejora la Agilidad.
	\item La simplicidad, o el arte de maximizar la cantidad de	trabajo no realizado, es esencial.
	\item Las mejores arquitecturas, requisitos y diseños emergen de equipos auto-organizados.
	\item A intervalos regulares el equipo reflexiona sobre	cómo ser más efectivo para a continuación ajustar y perfeccionar su comportamiento en consecuencia.
\end{itemize}

\section{Personal Extreme Programming}
La programación extrema o \textit{eXtreme Programming} (XP) es una metodología de desarrollo de la ingeniería de software formulada por Kent Beck, autor del primer libro sobre la materia, \textit{Extreme Programming Explained: Embrace Change} (1999). Es el más destacado de los procesos ágiles de desarrollo de software. Al igual que éstos, la programación extrema se diferencia de las metodologías tradicionales principalmente en que pone más énfasis en la adaptabilidad que en la previsibilidad.

Los defensores de XP consideran que los cambios de requisitos sobre la marcha son un aspecto natural, inevitable e incluso deseable del desarrollo de proyectos. Creen que ser capaz de adaptarse a los cambios de requisitos en cualquier punto de la vida del proyecto es una aproximación mejor y más realista que intentar definir todos los requisitos al comienzo del proyecto e invertir esfuerzos después en controlar los cambios en los requisitos.

Se puede considerar la programación extrema como la adopción de las mejores metodologías de desarrollo de acuerdo a lo que se pretende llevar a cabo con el proyecto, y aplicarlo de manera dinámica durante el ciclo de vida del software.

\subsection{Características}
Las características fundamentales del método son:

\begin{itemize}
	\itemsep1pt \parskip1pt \parsep1pt
	\item Desarrollo iterativo e incremental: pequeñas mejoras, unas tras otras.
	\item Pruebas unitarias continuas, frecuentemente repetidas y automatizadas, incluyendo pruebas de regresión. Se aconseja escribir el código de la prueba antes de la codificación. Véase, por ejemplo, las herramientas de prueba JUnit orientada a Java, DUnit orientada a Delphi, NUnit para la plataforma.NET o PHPUnit para PHP. Estas tres últimas inspiradas en JUnit, la cual, a su vez, se inspiró en SUnit, el primer framework orientado a realizar tests, realizado para el lenguaje de programación Smalltalk.
	\item Programación en parejas: se recomienda que las tareas de desarrollo se lleven a cabo por dos personas en un mismo puesto. La mayor calidad del código escrito de esta manera -el código es revisado y discutido mientras se escribe- es más importante que la posible pérdida de productividad inmediata. En este sentido, la modalidad personal difiere por razones obvias; sin embargo, las demás características son perfectamente aplicables.
	\item Frecuente integración del equipo de programación con el cliente o usuario. Se recomienda que un representante del cliente trabaje junto al equipo de desarrollo.
	\item Corrección de todos los errores antes de añadir nueva funcionalidad. Hacer entregas frecuentes.
	\item Refactorización del código, es decir, reescribir ciertas partes del código para aumentar su legibilidad y mantenibilidad pero sin modificar su comportamiento. Las pruebas han de garantizar que en la refactorización no se ha introducido ningún fallo.
	\item Propiedad del código compartida: en vez de dividir la responsabilidad en el desarrollo de cada módulo en grupos de trabajo distintos, este método promueve el que todo el personal pueda corregir y extender cualquier parte del proyecto. Las frecuentes pruebas de regresión garantizan que los posibles errores serán detectados.
	\item Simplicidad en el código: es la mejor manera de que las cosas funcionen. Cuando todo funcione se podrá añadir funcionalidad si es necesario. La programación extrema apuesta que es más sencillo hacer algo simple y tener un poco de trabajo extra para cambiarlo si se requiere, que realizar algo complicado y quizás nunca utilizarlo.
\end{itemize}

La simplicidad y la comunicación son extraordinariamente complementarias. Con más comunicación resulta más fácil identificar qué se debe y qué no se debe hacer. Cuanto más simple es el sistema, menos tendrá que comunicar sobre éste, lo que lleva a una comunicación más completa, especialmente si se puede reducir el equipo de programadores.

\section{Kanban}

Kanban, del japonés \begin{CJK*}{UTF8}{gbsn}かんばん\end{CJK*}(\begin{CJK*}{UTF8}{gbsn}看板\end{CJK*}), que es literalmente ``letrero'' o ``cartelera'', es un sistema de programación para la producción \textit{Lean} y justo-a-tiempo.

Kanban se basa en una idea muy simple: el trabajo en curso (\textit{Work In Progress}, WIP) debería limitarse, y sólo deberíamos empezar con algo nuevo cuando un bloque de trabajo anterior haya sido entregado o ha pasado a otra función posterior de la cadena. El Kanban (o tarjeta señalizadora) implica que se genera una señal visual para indicar que hay nuevos bloques de trabajo que pueden ser comenzados porque el trabajo en curso actual no alcanza el máximo acordado.

Kanban usa un mecanismo de control visual para hacer seguimiento del trabajo conforme este viaja a través del flujo de valor. Típicamente, se usa un panel o pizarra con notas adhesivas o un panel electrónico de tarjetas. Las mejores prácticas apuntan probablemente al uso de ambos.

Las metodologías Ágiles han obtenido buenos resultados proporcionando transparencia respecto al trabajo en curso y completado, así como en el reporte de métricas como la velocidad (cantidad de trabajo realizada en una iteración). Kanban sin embargo va un paso más allá y proporciona transparencia al proceso y su flujo. Kanban expone los cuellos de botella, colas, variabilidad y desperdicios. Todas las cosas que impactan al rendimiento de la organización en términos de la cantidad de trabajo entregado y el ciclo de tiempo requerido para entregarlo. Kanban proporciona a los miembros del equipo y a las partes interesadas visibilidad sobre los efectos de sus acciones (o falta de acción). De esta forma, los casos de estudios preliminares están demostrando que Kanban cambia el comportamiento y motiva a una mayor colaboración en el trabajo. La visibilidad de los cuellos de botella, desperdicios y variabilidades y su impacto también promueve la discusión sobre la posibles mejoras, y los equipos comienzan rápidamente a implementar mejoras en su proceso.\cite{Skarin201003}

\subsection{Trello}

Trello es una plataforma de software que utiliza el paradigma Kanban para el manejo de proyectos. Iniciado en el año 2011 por la compañía Fog Creek Software y caracterizado como software de productividad, este utiliza un sistema de tableros, listas y tarjetas para llevar el control de múltiples tipos de proyecto, sin estar limitado a proyectos de desarrollo de software, por lo cual es altamente versátil. Las tarjetas aceptan comentarios, archivos adjuntos, votos, fechas de entrega y listas de verificación.

Para este proyecto, se abrió un tablero en Trello accesible desde la dirección \url{https://trello.com/b/yIMdcTCR/tesis-ros-kinect}, cuyas listas representan un híbrido entre las disponibles en PXP y Kanban, de la siguiente manera:

\begin{itemize}
	\itemsep1pt \parskip1pt \parsep1pt
	\item Pendientes: Tareas que están por ser ejecutadas.
	\item Actuales: Tareas que están siendo ejecutadas actualmente.
	\item Entregadas: Tareas que están marcadas como realizadas, pero que no han sido verificadas aún.
	\item Rechazadas: Tareas que, habiendo sido entregadas, presentan alguna falla o requieren algún cambio, por lo cual se colocan en esta lista para ser revisadas.
	\item Listas: Tareas que han sido verificadas y aceptadas.
	\item Ideas: Tarjetas que representan ideas aplicar, o tareas que no pueden pasar a la lista ``Pendientes'' por no haber recursos disponibles para ellas.
\end{itemize}

\begin{figure}[ht]
\centering
\includegraphics[width=1.00\textwidth]{trelloprojectwindow}
\caption{Tablero en Trello para el proyecto de grado}
\end{figure}

\subsubsection{Uso}

Cada tarea a realizar se coloca como una tarjeta en la lista correspondiente a ``Pendientes'' o ``Ideas'' según sea el caso, y una vez estén disponibles los recursos para tomarla (siendo estos recursos tiempo, o disponibilidad de la o las personas involucradas) se pasa dicha tarjeta, mediante arrastrar y colocar, a la lista de ``Actuales''. Esto permite ver el flujo de trabajo, tal como se mencionaba anteriormente, y permite saber el estado actual del proyecto si la tabla se mantiene actualizada.

Una vez terminada la tarea, se coloca en la lista ``Entregadas'', para su revisión (en el caso de este proyecto, el cliente del mismo viene a ser el profesor tutor, quien es la persona que puede verificar que se haya llevado a cabo satisfactoriamente o no). En caso que la tarea esté correcta, se puede pasar a la lista ``Listas'', lo cual marca la finalización de la tarea, y la liberación de los recursos empleados para tomar nuevas tareas disponibles. En caso de observar algún error, la tarea pasa a la lista ``Rechazadas'', para su revisión y corrección.
 % Marco teórico

\chapter{Marco Metodológico}

En este capítulo se pretende ahondar en la metodología utilizada para resolver el problema planteado desde el punto de vista de la gestión de proyectos, así como los cronogramas y métodos planteados para su solución.


\section{Desarrollo Orientado a Funcionalidades (FDD)}

La metodología elegida para la construcción de la plataforma planteada como problema en el Capítulo 1 fue el Desarrollo Orientado a Funcionalidades (FDD, por sus siglas en inglés) adaptado a una serie de entregas semanales, que sería reportadas los días sábado, desde el inicio del proceso de construcción de la plataforma, hasta el día de entrega, en la semana dieciséis (16) de desarrollo. Esta metodología fue elegida para facilitar el uso inmediato del sistema con sus funcionalidades actuales y, de esta manera, agilizar el proceso de pruebas de sistema, al realizarse en cada entrega.

El Desarrollo Orientado a Funcionalidades (Feature Driven Development o FDD) es uno de los llamados métodos ágiles dentro de la ingeniería del software, conocido por su alta adaptabilidad a los cambios y por sus tiempos de iteración cortos. Dicho método fue desarrollado por Jeff De Luca y Peter Code, y fue nombrado de esta manera en 1997 \cite{fdd}.
Goyal (2008) menciona que, en este enfoque, se asume que “cada etapa está 100\% completa antes de que la siguiente etapa comience” \cite{goyal2008}.
Por su parte, Khramtchenko (2004) destaca dos etapas principales de este proceso de desarrollo \cite{khramtchenko2004comparing}:
\begin{itemize}
    \item Descubrir la lista de funcionalidades a implementar.
    \item Implementación funcionalidad por funcionalidad.
\end{itemize}

Cada iteración dura alrededor de una a tres semanas, y al final de cada una de ellas se debe entregar una funcionalidad con la que el cliente pueda interactuar. Cada iteración, como se muestra también en la Figura \ref{figure:FDDworkflow} incluye el siguiente cuerpo formal:

\begin{itemize}
    \item Reunión inicial donde se especifican las funcionalidades requeridas.
    \item Proceso de diseño de clases, métodos y documentación.
    \item Revisión del diseño.
    \item Desarrollo de los métodos y las pruebas unitarias.
    \item Revisión del código.
    \item Despliegue de la funcionalidad.
\end{itemize}


\begin{figure}[H]
\centering
\includegraphics[width=0.80\textwidth]{img/5.png}
\caption{Plan de trabajo estándar del Desarrollo Orientado a Funcionalidades. Fuente: https://www.intellectsoft.net/blog/wp-content/uploads/Feature-Driven-Development-.jpg }
\label{figure:FDDworkflow}
\end{figure}


\section{Metodología FDD utilizada}

Para desglosar y sistematizar la metodología usada en el desarrollo de las funcionalidades de este producto, se procedió a hacer uso de un modelo propuesto por la Dynamic Domain Corporation \cite{fdd}. La misma divide la construcción del proyecto en cinco fases (entre paréntesis el porcentaje total de tiempo del proyecto invertido en cada fase), como se ve en la figura \ref{figure:FDDsteps}:

\begin{itemize}
    \item Desarrollar un modelo general. (14\%)
    \item Construir una lista de funcionalidades. (5\%)
    \item Planificar por funcionalidad. (4\%)
    \item Diseñar por funcionalidad.
    \item Desarrollar la funcionalidad. (77\%  tanto para diseño y desarrollo)

\end{itemize}

\begin{figure}[H]
\centering
\includegraphics[width=0.80\textwidth]{img/6.png}
\caption{Pasos usados en la metodología FDD descrita por la Dynamic Domain Corporation (Fuente: http://thedynamicdomain.com/content/images/feature-driven.jpg)}
\label{figure:FDDsteps}
\end{figure}

En ese sentido, se postuló el siguiente cronograma de ejecución. Los requerimientos específicos serán detallados en el capítulo siguiente:

\subsection{Semana 1 a Semana 2: Modelo General}

\begin{itemize}
    \item Reunión inicial con el CEO de Ignis Gravitas para definir los objetivos iniciales del producto.
    \item Reuniones continuas para obtener los planos esquemáticos visuales de la arquitectura de la información (conocidos comúnmente como wireframes) así como para su validación.
    \item Reunión final para validación de los wireframes de la primera iteración.
    \item Obtención del modelo visual general de la aplicación.
\end{itemize}

\subsection{Semana 3: Lista de Funcionalidades y Especificaciones Generales}
\begin{itemize}
    \item Desarrollo de lista de requerimientos funcionales del producto a construir.
    \item Depuración de los diseños iniciales de las interfaces visuales.
    \item Generación del diagrama de despliegue.
    \item Generación del diagrama del diseño de datos.
\end{itemize}

\subsection{Semana 4 y 5: Autenticación}

\begin{itemize}
    \item Investigación sobre las herramientas disponibles y estándares de seguridad en autenticaciones.
    \item Diseño del diagrama de casos de uso para la autenticación.
    \item Implementación funcional de la API de autenticación.
    \item Implementación de las interfaces gráficas de usuario para el proceso de autenticación.
    \item Pruebas unitarias sobre la autenticación.
\end{itemize}

\subsection{Semana 6 y 7:  Publicaciones}

\begin{itemize}
    \item Diseño de diagramas de uso de las distintas publicaciones del sitio.
    \item Construcción de la vista principal de publicaciones.
    \item Definición y adaptación de la vista principal de búsquedas.
    \item Desarrollo de la interfaz gráfica para las publicaciones.
    \item Implementación de la API de publicaciones.
Pruebas unitarias y de integración sobre las publicaciones
\end{itemize}

\subsection{Semanas 8 y 9: Perfiles}

\begin{itemize}
    \item Construcción de diagramas de uso de los distintos perfiles del sitio.
    \item Construcción de las interfaces gráficas para los distintos perfiles especificados en las reuniones.
    \item Implementación de la API de perfiles.
    \item Pruebas unitarias y de integración sobre cada uno de los perfiles.
\end{itemize}

\subsection{Semana 10 y 11: Trabajos}

\begin{itemize}
    \item Construcción de diagramas de uso de la sección de búsquedas y muestra de trabajos.
    \item Construcción de las interfaces gráficas de la sección de trabajos.
    \item Implementación de la API de trabajos.
    \item Pruebas unitarias y de integración sobre la sección de trabajos.
\end{itemize}

\subsection{Semana 12 y 13: Productos}

\begin{itemize}
    \item Diseño de diagramas de uso de la sección de búsquedas y muestra de productos.
    \item Construcción de las interfaces de usuario de la galería de productos y propuestas de valor.
    \item Implementación de la API de productos.
    \item Pruebas unitarias y de integración sobre la vista de productos.
\end{itemize}

\subsection{Semanas 14 y 15: Pruebas de sistema}

\begin{itemize}
    \item Despliegue completo de la primera versión de la aplicación.
    \item Correcciones menores en reuniones constantes con los interesados del producto.
    \item Pruebas de sistema completas sobre toda la plataforma.
    \item Optimizaciones de pertinencia sobre la capa de interfaces y la capa de datos.
\end{itemize}

\subsection{Semana 16: Entrega del Producto}

\begin{itemize}
    \item Entrega final del producto.
    \item Planificación para mantenimiento del producto.
\end{itemize} % Análisis metodológico

\chapter{ROS}

En este capítulo se desea profundizar en la definición de ROS (\textit{Robot Operating System}) como plataforma de software seleccionada para el desarrollo de este proyecto de grado. Asimismo, se describen sus características principales, la arquitectura de software que utiliza, la instalación del mismo en el entorno de desarrollo a utilizar, y los módulos en los cuales se apoya este proyecto para llevar a cabo la generación de mapas de entorno.

\section{Definición}

\textit{Robot Operating System} (Sistema Operativo de/para Robot) o sencillamente ROS es, tal como su nombre implica, un sistema operativo para robots, de forma similar a los sistemas operativos para computadores de escritorio o servidores. Desarrollado y mantenido por la empresa Willow Garage desde 2008 hasta 2013, siendo tomada su dirección en ese año por la Fundación de Robótica de Código Abierto, es una colección de herramientas, bibliotecas y convenciones que buscan simplificar la tarea de crear comportamientos de robot robustos y complejos a lo largo de una amplia variedad de plataformas robóticas.

La justificación de porqué hacer esto es porque decididamente, crear software robótico de propósito general y verdaderamente robusto es difícil, ya que si bien para un ser humano algunos problemas son triviales, no lo son en lo absoluto al momento de tomar en cuenta las grandes variaciones entre instancias de tareas y entornos. Lidiar con estas variaciones es tan complicado que ningún individuo, laboratorio o institución pudiera esperar llevarlo a cabo por su propia cuenta.

Por ello, ROS fue construido desde cero con el fin de alentar el desarrollo de software robótico de forma colaborativa. Un ejemplo de esto es que, un laboratorio podría tener expertos en cartografía o mapeado de interiores y podría contribuir un sistema de excelente calidad para la producción de mapas. Otro grupo podría tener expertos en el uso de mapas para navegar, y otro grupo podría haber descubierto un enfoque de visión por computador que funciona bien para el reconocimiento de objetos pequeños entre el desorden. ROS fue diseñado específicamente para grupos como éstos para colaborar y construir sobre el trabajo del otro. \cite{aboutros}

Además, ROS es Software Libre y está distribuido bajo la licencia BSD, permitiendo el desarrollo de proyectos comerciales y no-comerciales. Una característica importante en cuanto a la arquitectura (que se detallará más adelante) es que ROS funciona a través de comunicación entre procesos, sin requerir que los módulos sean enlazados dentro del mismo ejecutable, por lo que cualquier sistema construido usando ROS como base puede tener control detallado sobre las licencias de software que utilicen sus módulos, ya sean GPL, BSD o cualquier otra hasta propietaria. \cite{quigley2009ros}

\section{Características Principales}

Se pueden comentar las siguientes:

\begin{description}
	\item[Comunicación entre pares:] los sistemas robóticos complejos con múltiples enlaces podrían tener varios computadores de a bordo (para realizar tareas paralelas) conectados a través de una red. La ejecución de un maestro central podría dar lugar a la congestión severa en un enlace determinado. Usando una comunicación peer-to-peer o entre pares evitaría este problema. En ROS, una arquitectura peer-to-peer acoplado a un sistema de memoria intermedia o \textit{buffer} y un sistema de búsqueda (un servicio de nombres llamado ``maestro'' en ROS), le permite a cada componente dialogar directamente con cualquier otro, de forma sincrónica o asincrónica como sea necesario.

	\item[Gratuito y de código abierto:] Ser una plataforma de código abierto ofrece la reutilización de funciones ya existentes proporcionadas por muchos otros usuarios de ROS. Su código se suministra en repositorios como \textit{stacks}, o ``pilas''. Otras personas han desarrollado capacidades sorprendentes para los robots que han sido ``de código abierto'' y son relativamente fáciles de añadir de forma incremental utilizando el entorno de desarrollo de ROS.

	\item[Delgado:] Para combatir el desarrollo de algoritmos que se ``enredan'' o vinculan en un grado mayor o menor con el sistema operativo del robot y, por tanto, son difíciles de reutilizar posteriormente, los desarrolladores de ROS han planificado que los controladores y otros algoritmos sean contenidos en ejecutables independientes. Esto garantiza la máxima reutilización y, sobre todo, mantiene reducido su tamaño. Este método hace que ROS sea fácil de usar y ubica la complejidad en las bibliotecas. Esta disposición también facilita las pruebas unitarias y los sistemas desarrollados puede ser completamente independientes de otro sistema.

	\item[Multi-lenguaje:] ROS es independiente del lenguaje, y se puede programar en varios lenguajes. La especificación ROS trabaja en la capa de mensajería. Las conexiones \textit{peer-to-peer} se negocian en XML-RPC, que existe en un gran número de lenguajes. Para soportar un nuevo lenguaje, se pueden reenvolver clases C ++ (lo cual se hizo para el cliente Octave, por ejemplo) o se escriben clases permitiendo que se generen mensajes. Estos mensajes se describen en IDL (\textit{Interface Definition Language}). \cite{quigley2009ros}
\end{description}

\section{Arquitectura}

ROS está implementado bajo los siguientes conceptos fundamentales:

\begin{itemize}
	\itemsep1pt \parskip1pt \parsep1pt
	\item Nodos: Son procesos que realizan cálculos; en el contexto de ROS, este término es intercambiable con ``módulo de software'' ya que está diseñado para ser altamente modular: un sistema está compuesto típicamente de muchos nodos.

	\item Mensajes:	Los nodos se comunican entre si al pasar mensajes, que no es más que una estructura de datos de tipo estricto. Los tipos de datos soportados pueden ser estándar (entero, flotante, booleano, etc.), así como también arreglos de estos o constantes. Un mensaje puede estar compuesto por varios mensajes y el nivel de anidamiento al que pueden llegar es arbitrario.

	\item Tópicos: Un nodo publica un mensaje a través de un tópico, que es sencillamente una cadena de caracteres tal como ``odometría'' o ``mapa''. Un nodo que esté interesado en un tipo de dato específico se suscribirá al tópico apropiado. En cualquier momento dado, pueden existir múltiples publicadores o suscriptores de forma concurrente para un tópico particular y un nodo puede publicar o suscribirse a múltiples tópicos. Por lo general, los publicadores y suscriptores no están al tanto de la existencia del otro.

	\item Servicios: Si bien el modelo publicar-suscribir basado en tópicos es un paradigma de comunicaciones flexible, el esquema de enrutamiento de ``emisión'' no es apropiado para las transacciones síncronas, lo cual puede simplificar el diseño de algunos nodos. A esto se le llama ``servicio'' en ROS, definidos por un nombre y un par de mensajes tipados, uno para la petición y otro para la respuesta. Es de notar que, a diferencia de los tópicos, solo un nodo puede anunciar un servicio con un nombre particular; por ejemplo, solamente puede haber un servicio llamado ``clasificar\_imagen''. \cite{quigley2009ros}
\end{itemize}

\section{Requisitos de Instalación}

ROS está organizado en distribuciones, que cuentan cada una con sus respectivos requisitos de instalación. Por consenso, cada mes de mayo se lanza una nueva distribución de ROS, y las distribuciones de años pares son de soporte extendido, con 5 años de soporte. Las distribuciones de años impares son distribuciones regulares, con soporte por 2 años. Igualmente, la plataforma soportada por defecto es Ubuntu Linux, mientras que otros sistemas operativos, tales como OS X, Android, Arch Linux, Debian y Windows están bajo soporte experimental, por parte de la comunidad.

Al momento de la elaboración de este proyecto, la versión instalada es ``\textit{Indigo Igloo}'' (con soporte LTS o \textit{Long Term Support} -- Soporte de Larga Duración), con los siguientes requisitos:

\begin{itemize}
	\itemsep1pt \parskip1pt \parsep1pt
	\item Ubuntu Saucy (13.10) o Ubuntu Trusty (14.04 LTS)
	\item C++03
	\item Boost 1.53
	\item Lisp SBCL 1.0.x
	\item Python 2.7
	\item CMake 2.8.11 \cite{rosrequirements}
\end{itemize}

Durante el proceso de instalación, se lleva a cabo la satisfacción de dependencias.

\section{Procedimiento de Instalación}

\subsection{Descripción de Entornos de Desarrollo}

Los entornos y dispositivos utilizados para llevar a cabo el proyecto, para pruebas de instalación y/o funcionamiento, son los siguientes:

\begin{itemize}
	\item Máquina Virtual: Oracle \textregistered{} VirtualBox VM\textsuperscript{TM}, 2 GB RAM, 30 GB disco duro, con Ubuntu Trusty Tahr 14.04.2 de 64 bits.
	\item Computador Portatil: Lenovo \textregistered{} Z50-70, procesador Intel \textregistered{}  Core\textsuperscript{TM} i7-4510U, 6 GB RAM, 500 GB disco duro, con Ubuntu Trusty Tahr 14.04.2 de 64 bits.
	\item Microsoft \textregistered{}  XBOX 360 \textregistered{}  Kinect\textsuperscript{TM}
\end{itemize}

En la máquina virtual se llevó a cabo la comprobación del procedimiento de instalación, ya que es un entorno que permite restaurar a un punto anterior con facilidad, en caso de inconvenientes tales como paquetes mal instalados, etc.

\subsection{Instalación}

El proceso de instalación de ROS está excelentemente documentado en su \textit{wiki} oficial accesible desde \url{http://wiki.ros.org/ROS/Installation}, por lo cual seguimos los pasos tomando nota de cualquier dependencia faltante o error.

Para comenzar, en el enlace previamente mencionado, hacemos clic en ``Indigo installation instructions'', puesto que es la versión de soporte extendido y es la compatible con la versión instalada de Ubuntu.

Una vez allí, bajo el apartado ``Supported'', hacemos clic en ``Ubuntu''.

En adelante, solamente seguimos los siguientes pasos haciendo énfasis en la versión instalada de Ubuntu, tomados directamente del sitio:

\begin{enumerate}
\renewcommand{\labelenumii}{\theenumii}
\renewcommand{\theenumii}{\theenumi.\arabic{enumii}.}
	\item Configure sus repositorios de Ubuntu

	Configure sus repositorios de Ubuntu para activar los repositorios ``restricted'', ``universe'' y ``multiverse''. La guía de configuración de repositorios de Ubuntu, disponible en el siguiente enlace \url{https://help.ubuntu.com/community/Repositories/Ubuntu} (en inglés) detalla adecuadamente los pasos necesarios.
	\item Configure el archivo sources.list

	Configure su computador para aceptar software desde packages.ros.org. Esta distribución de ROS \textbf{sólo} soporta Saucy (Ubuntu 13.10) y Trusty (Ubuntu 14.04) para paquetes Debian.

	\begin{blackcodebox}
	\begin{lstlisting}[language=bash]
sudo sh -c 'echo "deb http://packages.ros.org/ros/ubuntu $(lsb_release -sc) main" > /etc/apt/sources.list.d/ros-latest.list'
	\end{lstlisting}
	\end{blackcodebox}

	\item Configure sus llaves

	\begin{blackcodebox}
	\begin{lstlisting}[language=bash]
sudo apt-key adv --keyserver hkp://pool.sks-keyservers.net --recv-key\\ 0xB01FA116
	\end{lstlisting}
	\end{blackcodebox}

	\item Instalación
	Primero debe asegurarse que su índice de paquetes Debian esté actualizado:

	\begin{blackcodebox}
	\begin{lstlisting}[language=bash]
sudo apt-get update
	\end{lstlisting}
	\end{blackcodebox}

	Si está utilizando Ubuntu Trusty 14.04.2 y experimenta problemas con las dependencias durante la instalación de ROS, quizás deba instalar dependencias adicionales del sistema.

	\begin{redwarningbox}
	No instale estos paquetes si está utilizando 14.04, ya que destruirá su servidor X (gráfico):
	\end{redwarningbox}
	\begin{blackcodebox}
	\begin{lstlisting}[language=bash]
sudo apt-get install xserver-xorg-dev-lts-utopic mesa-common-dev-lts-utopic libxatracker-dev-lts-utopic libopenvg1-mesa-dev-lts-utopic libgles2-mesa-dev-lts-utopic libgles1-mesa-dev-lts-utopic libgl1-mesa-dev-lts-utopic libgbm-dev-lts-utopic libegl1-mesa-dev-lts-utopic
	\end{lstlisting}
	\end{blackcodebox}
	\begin{redwarningbox}
	No instale los paquetes anteriores si está utilizando 14.04, ya que destruirá su servidor X (gráfico). Alternativamente, intente instalar sólo lo siguiente para corregir problemas de dependencias:
	\end{redwarningbox}
	\begin{blackcodebox}
	\begin{lstlisting}[language=bash]
sudo apt-get install libgl1-mesa-dev-lts-utopic
	\end{lstlisting}
	\end{blackcodebox}

	Existen muchas bibliotecas y herramientas en ROS. Se han provisto cuatro configuraciones por defecto para iniciar. También se pueden instalar paquetes de ROS de forma individual.

	\begin{itemize}
		\item Instalación de Escritorio Completa: (Recomendada): ROS, rqt, rviz, bibliotecas genéricas para robots, simuladores 2D/3D, navegación y percepción 2D/3D

		Indigo usa Gazebo 2, la cual es la versión por defecto en Trusty y es la recomendada. Si desea actualizar a Gazebo 3 vea las instrucciones en \url{http://wiki.gazebosim.org/wiki/Install/Gazebo_and_ROS#Gazebo_3.x_series} acerca de cómo actualizar el simulador.


		\begin{blackcodebox}
		\begin{lstlisting}[language=bash]
sudo apt-get install ros-indigo-desktop-full
		\end{lstlisting}
		\end{blackcodebox}

		\item Instalación de Escritorio: ROS, rqt, rviz, y bibliotecas genéricas para robots.

		\begin{blackcodebox}
		\begin{lstlisting}[language=bash]
sudo apt-get install ros-indigo-desktop
		\end{lstlisting}
		\end{blackcodebox}

		\item ROS-Base: (Esencial) Las bibliotecas de paquete, generación y comunicación. No incluye herramientas de entorno gráfico.

		\begin{blackcodebox}
		\begin{lstlisting}[language=bash]
sudo apt-get install ros-indigo-ros-base
		\end{lstlisting}
		\end{blackcodebox}

		\item Paquete Individual: También puede instalar un paquete específico de ROS package (reemplace subguiones con guiones del nombre del paquete):

		\begin{blackcodebox}
		\begin{lstlisting}[language=bash]
sudo apt-get install ros-indigo-PAQUETE
		\end{lstlisting}
		\end{blackcodebox}
		por ejemplo:

		\begin{blackcodebox}
		\begin{lstlisting}[language=bash]
sudo apt-get install ros-indigo-slam-gmapping
		\end{lstlisting}
		\end{blackcodebox}

		Para listar paquetes disponibles, use:

		\begin{blackcodebox}
		\begin{lstlisting}[language=bash]
apt-cache search ros-indigo
		\end{lstlisting}
		\end{blackcodebox}
	\end{itemize}
	\item Inicialice rosdep

	Antes de poder utilizar ROS, se debe inicializar rosdep. rosdep permite instalar dependencias del sistema con facilidad para código fuente que desee compilar y es requerido para poder ejecutar algunos componentes centrales en ROS.

	\begin{blackcodebox}
	\begin{lstlisting}[language=bash]
sudo rosdep init
rosdep update
	\end{lstlisting}
	\end{blackcodebox}

	\item Configuración de entorno

	Es conveniente si las variables de entorno de ROS son agregadas automáticamente a su sesión Bash cada vez que se invoca un nuevo terminal:

	\begin{blackcodebox}
	\begin{lstlisting}[language=bash]
echo "source /opt/ros/indigo/setup.bash" >> ~/.bashrc
source ~/.bashrc
	\end{lstlisting}
	\end{blackcodebox}

	Si tiene más de una distribución de ROS instalada, \url{~/.bashrc} sólo debe tomar como fuente el \url{setup.bash} para la versión que esté en uso actualmente.

	Si simplemente desea cambiar el entorno del terminal actual, puede escribir:

	\begin{blackcodebox}
	\begin{lstlisting}[language=bash]
source /opt/ros/indigo/setup.bash
	\end{lstlisting}
	\end{blackcodebox}

	\item Obtener rosinstall

rosinstall es una herramienta de línea de comandos frecuentemente utilizada en ROS que es distribuida por separado. Le permite descargar con facilidad muchos árboles fuentes para paquetes ROS con un solo comando.

Para instalar esta herramienta en Ubuntu, ejecute:

	\begin{blackcodebox}
	\begin{lstlisting}[language=bash]
sudo apt-get install python-rosinstall
	\end{lstlisting}
	\end{blackcodebox}
\end{enumerate}

\section{Módulos Disponibles para SLAM en ROS}

Debido a la naturaleza propia del software, es improbable, para no decir imposible, poder realizar un listado exhaustivo de todos los módulos disponibles. Sin embargo, se listan acá los más populares, con el fin de brindar alternativas para elaborar mapas de entorno con distintos tipos de sensores.

Es de notar que algunos paquetes requieren obtener datos de un sensor láser, por lo cual el Kinect no cumpliría con ese requerimiento; sin embargo, hay un paquete o nodo de ROS que permite convertir las imágenes con datos de profundidad capturadas por un sensor RGB-D (tal como el Kinect) y emular un sensor láser. Este puede encontrarse acá: \url{http://wiki.ros.org/depthimage_to_laserscan}

\begin{itemize}
	\itemsep1pt \parskip1pt \parsep1pt
	\item gmapping (\url{http://wiki.ros.org/gmapping})
	\item hector\_slam (\url{http://wiki.ros.org/hector_slam})
	\item RTAB-Map (\url{http://wiki.ros.org/rtabmap})
	\item rgbdslam (\url{http://wiki.ros.org/rgbdslam})
	\item rgbdslam V2 (versión actualizada) (\url{http://felixendres.github.io/rgbdslam_v2/})
\end{itemize}

Tras evaluar documentación disponible, ejemplos de código y funcionamiento y versiones soportadas de cada módulo en ROS, se decidió utilizar RTAB-Map (descrito en el siguiente capítulo).

Es de destacar, que RTAB-Map cuenta con soporte directo por parte del desarrollador del mismo, por lo cual será utilizado en este proyecto para la generación de mapas. % Análisis de requerimientos

\chapter{Generación de Mapas a través de RTAB-Map}

En este capítulo se describe el software RTAB-Map, sus características, su proceso de funcionamiento, su instalación y uso en el entorno de desarrollo en sus presentaciones como software independiente o como módulo integrado a ROS y por último, la generación de un mapa de entorno a través del mismo.

\section{Definición}

RTAB-Map (\textit{Real-Time Appearance-Based Mapping} o Cartografía en Tiempo Real Basada en Apariencias) es una aproximación a SLAM mediante Grafo RGB-D basado en un detector de cierre de lazo Bayesiano global. El detector de cierre de lazo usa un modelo de bolsa de palabras para determinar la probabilidad de que una nueva imagen provenga de una ubicación anterior o nueva. Cuando una hipótesis de cierre de lazo es aceptada, una nueva restricción es agregada al grafo del mapa, y un optimizador de grafo minimiza los errores en el mapa. \cite{labbe14online}

Un enfoque de manejo de memoria es utilizado para limitar el número de ubicaciones utilizadas para la detección de cierre de lazos y la optimización del grafo \cite{labbe13appearance}, para que las restricciones de tiempo real en entornos de gran escala sean siempre respetadas. RTAB-Map puede ser utilizado por si solo con un Kinect o cámara estéreo operado a mano para obtener cartografía RGB-D de 6 grados de libertad, o en un robot equipado con un medidor de distancias láser para cartografía de 3 grados de libertad. \cite{rtabmaphome}

La aplicación soporta distintos sensores, tales como el Kinect, el ASUS Xtion Pro / Pro Live \cite{xtionpro} \cite{xtionprolive}, cámaras soportadas por la biblioteca libdc1394 \cite{libdc1394} y cámaras soportadas por la biblioteca FlyCapture2. \cite{libflycapture2}

Como detalle adicional, RTAB-Map permite su uso directamente desde código C++, tanto para detección de cierre de lazos únicamente, como para generar mapas RGB-D.

\section{Instalación}

RTAB-Map soporta instalaciones en Linux, OS X y Windows y puede funcionar de dos maneras: como software independiente (no necesita otros paquetes de software aparte de él mismo y de los controladores de las cámaras a usar) o como un módulo de ROS, en cuyo caso lógicamente requiere que ROS esté instalado de antemano (y la compatibilidad con los distintos sistemas operativos se reduce a la misma de ROS).

\subsection{Como software independiente}

Para instalar el software de forma independiente, tomaremos las instrucciones correspondientes a la instalación en Ubuntu debido a que es el entorno de desarrollo utilizado. Estas instrucciones son tomadas del repositorio del proyecto en Github, a través de la dirección \url{https://github.com/introlab/rtabmap/wiki/Installation}:

\begin{itemize}
	\item Con ROS ya instalado en el sistema:

	Si ya está instalado ROS en el sistema (como es el caso en el desarrollo del proyecto), ya algunas dependencias estarán instaladas:

	Dependencias según versión de ROS:
	Indigo:
	\begin{blackcodebox}
	\begin{lstlisting}[language=bash]
sudo apt-get install libsqlite3-dev libpcl-1.7-all libfreenect-dev libopencv-dev
	\end{lstlisting}
	\end{blackcodebox}
	Hydro:
	\begin{blackcodebox}
	\begin{lstlisting}[language=bash]
sudo apt-get install libsqlite3-dev libpcl-1.7-all ros-hydro-libfreenect ros-hydro-opencv2
	\end{lstlisting}
	\end{blackcodebox}

	Descargue el código fuente de RTAB-Map desde Github:
	\begin{blackcodebox}
	\begin{lstlisting}[language=bash]
git clone https://github.com/introlab/rtabmap.git rtabmap
cd rtabmap/build
cmake ..
make -j4
make install
	\end{lstlisting}
	\end{blackcodebox}

	Ya puede ejecutar la aplicación (llamada ``rtabmap'').

	\item Si ROS no está instalado:

	Dependencias del sistema:
	\begin{blackcodebox}
	\begin{lstlisting}[language=bash]
sudo apt-get install libsqlite3-dev libpcl-1.7-all libopencv-dev
	\end{lstlisting}
	\end{blackcodebox}

	Para instalar libpcl-1.7-all, es posible que deba agregar los repositorios de ROS (en este caso particular, de la distribución de ROS compatible con la distribución de Ubuntu) realizando los siguientes pasos:
	\begin{blackcodebox}
	\begin{lstlisting}[language=bash]
sudo sh -c 'echo "deb http://packages.ros.org/ros/ubuntu $(lsb_release -sc) main" > /etc/apt/sources.list.d/ros-latest.list'
wget http://packages.ros.org/ros.key -O - | sudo apt-key add -
sudo apt-get update
	\end{lstlisting}
	\end{blackcodebox}

	Si desea habilitar las características SURF/SIFT (SURF: \textit{Speeded-Up Robust Features} -- Características Robustas Aceleradas; SIFT: \textit{Scale-Invariant Feature Transform} -- Transformación de Características Invariantes en Escala) en RTAB-Map, deberá descargar y generar OpenCV desde el código fuente para tener acceso al módulo no-libre/privativo:
	\begin{blackcodebox}
	\begin{lstlisting}[language=bash]
cd opencv
mkdir build
cd build
cmake -DCMAKE_BUILD_TYPE=Release ..
make -j4
sudo make install
	\end{lstlisting}
	\end{blackcodebox}

	Descargue el código fuente de RTAB-Map desde Github: obtenga la última versión o el código fuente actual
	\begin{blackcodebox}
	\begin{lstlisting}[language=bash]
git clone https://github.com/introlab/rtabmap.git rtabmap
cd rtabmap/build
cmake ..
make -j4
sudo make install
	\end{lstlisting}
	\end{blackcodebox}

	Ya puede ejecutar la aplicación (llamada ``rtabmap'').

	\item Actualizar el código fuente de RTAB-Map

	Si desea incorporar los últimos cambios después de realizar el ``git clone'' puede actualizarlo de la siguiente forma:
	\begin{blackcodebox}
	\begin{lstlisting}[language=bash]
cd rtabmap
git pull origin master
cd build
cmake ..
make -j4
sudo make install
	\end{lstlisting}
	\end{blackcodebox}
\end{itemize}

\subsection{Como módulo de ROS}

Ubuntu cuenta con binarios para las versiones Hydro e Indigo; basta con ejecutar el siguiente comando, según sea la distribución de ROS:

\noindent ROS Hydro:
\begin{blackcodebox}
\begin{lstlisting}[language=bash]
sudo apt-get install ros-hydro-rtabmap-ros
\end{lstlisting}
\end{blackcodebox}

\noindent ROS Indigo:
\begin{blackcodebox}
\begin{lstlisting}[language=bash]
sudo apt-get install ros-indigo-rtabmap-ros
\end{lstlisting}
\end{blackcodebox}

Si se desea instalar desde fuente, el proceso (detallado en la página \url{https://github.com/introlab/rtabmap_ros#rtabmap_ros}) conlleva tener conocimiento del espacio de trabajo (\textit{workspace}) de ROS, y la instalación desde fuente de la biblioteca OpenCV. También se asume que se ha configurado el espacio de trabajo en el directorio \url{~/catkin_ws} y que el archivo \url{~/.bashrc} contiene lo siguiente:

\noindent ROS Hydro:
\begin{blackcodebox}
\begin{lstlisting}[language=bash]
source /opt/ros/hydro/setup.bash
source ~/catkin_ws/devel/setup.bash
\end{lstlisting}
\end{blackcodebox}

\noindent ROS Indigo:
\begin{blackcodebox}
\begin{lstlisting}[language=bash]
source /opt/ros/indigo/setup.bash
source ~/catkin_ws/devel/setup.bash
\end{lstlisting}
\end{blackcodebox}

Luego, se procede a descargar el código fuente de RTAB-Map desde Github (\textbf{NOTA:} No descargar dentro del espacio de trabajo) e instalarlo dentro del directorio \url{devel} en el espacio de trabajo, ejecutando lo siguiente:
\begin{blackcodebox}
\begin{lstlisting}[language=bash]
cd ~
git clone https://github.com/introlab/rtabmap.git rtabmap
cd rtabmap/build
cmake -DCMAKE_INSTALL_PREFIX=~/catkin_ws/devel ..
make -j4
sudo make install
\end{lstlisting}
\end{blackcodebox}

Ahora puede instalar el ros-pkg de RTAB-Map dentro del directorio \url{src} del espacio de trabajo Catkin:
\begin{blackcodebox}
\begin{lstlisting}[language=bash]
cd ~/catkin_ws
git clone https://github.com/introlab/rtabmap_ros.git src/rtabmap_ros
catkin_make
\end{lstlisting}
\end{blackcodebox}

\section{Generación de Mapas de Entorno}
\label{sec:genenvironmentmaps}

Una vez instalado y ejecutado el software, podremos seleccionar un dispositivo entre la lista previamente mencionada, generar una base de datos y comenzar a reconocer el entorno.

RTAB-Map comenzará a capturar una nube de puntos en 3D mientras detecta cierres de lazo -- es decir, se lleva a cabo la detección en las ``imágenes'' (ya que no se captura una imagen en sí, sino puntos con datos de profundidad y color) de lugares ya visitados previamente, tras lo cual se realizan correcciones en las estimaciones pasadas. Una vez terminado, se puede exportar la nube de puntos capturada a distintos formatos (PCD, que es un formato que contiene datos de nube de puntos \cite{pcdformat} y PLY, que es un formato de polígonos, que guardan datos tridimensionales), para ser procesada posteriormente de ser necesario.

\subsection{Requerimientos}

Para funcionar de forma adecuada, RTAB-Map requiere el uso de las bibliotecas PCL y OpenCV, así como también de los controladores correspondientes al dispositivo que desee usarse, ya sea el Kinect, el Xtion o las múltiples opciones de cámaras estéreo. Asimismo, dependiendo de la configuración del robot y de los sensores disponibles, tendremos múltiples opciones para realizar la cartografía del entorno, todas estas detalladas en la documentación disponible de RTAB-Map, accesible desde el sitio web \url{http://wiki.ros.org/rtabmap_ros/Tutorials/SetupOnYourRobot} (en inglés).

\subsection{Procedimiento y Pruebas}

Con el fin de probar el funcionamiento de RTAB-Map como aplicación independiente e integrada a ROS, seguiremos los tutoriales disponibles en la documentación, primero generando una nube de puntos desde la aplicación independiente, y luego generándola y visualizándola desde ROS.

Cabe destacar nuevamente que, al no contar con un robot propiamente dicho, estaremos utilizando la opción de realizar el mapa únicamente con el Kinect.

\begin{figure}[b]
\centering
\includegraphics[width=0.75\textwidth]{freenecttest}
\caption{Comprobación de las cámaras del Kinect}
\label{figure:freenecttest}
\end{figure}

Para ello se ha cumplido tanto con la instalación independiente como con la instalación integrada a ROS y se ha comprobado el reconocimiento del Kinect a través del comando:

\begin{blackcodebox}
\begin{lstlisting}[language=bash]
freenect-glview
\end{lstlisting}
\end{blackcodebox}

\noindent el cual muestra las imágenes provenientes de ambas cámaras, como se muestra en la figura \ref{figure:freenecttest}

Una vez realizada esta comprobación, podemos continuar con la generación de un mapa de entorno 3D.

\subsubsection{Generación de mapas 3D}

Para esto, realizamos el reconocimiento del entorno a través de la aplicación independiente. Para ello, se inició un terminal y se escribió el comando:

\begin{blackcodebox}
\begin{lstlisting}[language=bash]
rtabmap
\end{lstlisting}
\end{blackcodebox}

\noindent cuyo binario está instalado en \url{/usr/local/bin/rtabmap}. Este procedimiento ejecuta el programa, el cual nos pregunta en la primera inicialización, dónde deseamos guardar los datos de configuración, así como los de la captura de datos (por defecto, esto se realiza en las carpetas \url{/home/<usuario>/.rtabmap/} y \url{/home/<usuario>/Documents/RTAB-Map}, respectivamente).

\begin{figure}[hb]
\centering
\raisebox{-0.5\height}{\includegraphics[width=0.40\textwidth]{rtabmapinit}}\hspace{1em}%
\raisebox{-0.5\height}{\includegraphics[width=0.40\textwidth]{rtabmapinitdb}}
\caption{Inicialización de RTAB-Map}
\label{figure:rtabmapinit}
\end{figure}

Tras iniciar, se nos presenta la siguiente pantalla:

\begin{figure}[H]
\centering
\includegraphics[width=0.70\textwidth,height=4.4cm]{rtabmapmain}
\caption{Ventana principal de RTAB-Map}
\label{figure:rtabmapmain}
\end{figure}

Para comenzar la captura de datos de entorno, inicializamos una nueva base de datos haciendo clic en el botón ``New database'':

\begin{figure}[H]
\centering
\includegraphics[width=0.40\textwidth]{rtabmapstartdb}
\caption{Barra de menú principal de RTAB-Map}
\label{figure:rtabmapstartdb}
\end{figure}

\noindent el cual genera las estructuras de datos necesarias y tras su culminación, estamos listos para realizar la captura: hacemos clic en el botón Iniciar (botón ``Reproducir'' en la figura \ref{figure:rtabmapstartdb}) y ya podemos comenzar a explorar el entorno. Simplemente basta con dirigir el Kinect haciendo un recorrido del área a obtener, mientras se supervisa la interfaz gráfica de RTAB-Map.

\begin{figure}[H]
\centering
\includegraphics[width=0.70\textwidth]{rtabmapstandalonebedroommap}
\caption{Generación de un mapa de una habitación}
\label{figure:rtabmapstandalonebedroommap}
\end{figure}

Si en algún momento de la exploración, la pantalla de ``\textit{odometry}'' (odometría) muestra un fondo rojo junto con una imagen estática del entorno, quiere decir que no se tienen suficientes características distintivas en la zona capturada para poder realizar una medición adecuada. Para arreglarlo, tal como se nos indica en el tutorial correspondiente, debemos colocar la cámara (el Kinect) de nuevo en la zona que se muestra en la imagen y esperar a que la aplicación retome la generación del mapa desde ese punto.

Una vez explorado el área, hacemos clic en el botón ``Stop'', tras lo cual podemos examinar los resultados, presentados en la figura \ref{figure:bedroommap}.

\begin{figure}[H]
\centering
\includegraphics[width=0.80\textwidth]{bedroommap}
\caption{Generación de nube de puntos de una habitación con RTAB-Map}
\label{figure:bedroommap}
\end{figure}

En el mapa 3D generado, se puede observar una serie de puntos en cian conectados entre sí, que representan la trayectoria de la cámara (en este caso, del Kinect), al generar el mapa.

\subsubsection{Generación de mapas 2D}

La generación de un mapa 2D no es sino la proyección en el plano de la nube de puntos capturada en el paso anterior. Para esto, utilizaremos la versión de RTAB-Map integrada a ROS, ya que por inconvenientes al momento de instalar la versión independiente, no se reconoce de forma adecuada la biblioteca PCL, necesaria para generar la ``rejilla de ocupación'' (en inglés, ``\textit{occupancy grid}'') que sería el mapa en 2D de los obstáculos detectados durante la generación del mapa.

Procedemos de la siguiente forma:

En primer lugar, requeriremos el uso del software de control para el Turtlebot \cite{turtlebot}, que es un kit robótico de bajo costo. Este software permite simular un robot y provee de archivos de lanzamiento adecuados para nuestro caso de uso. Se instala mediante el siguiente comando:

\begin{blackcodebox}
\begin{lstlisting}[language=bash]
sudo apt-get install ros-indigo-turtlebot ros-indigo-turtlebot-apps ros-indigo-turtlebot-interactions ros-indigo-turtlebot-simulator ros-indigo-kobuki-ftdi ros-indigo-rocon-remocon ros-indigo-rocon-qt-library ros-indigo-ar-track-alvar-msgs
\end{lstlisting}
\end{blackcodebox}

Una vez instalado, debemos agregar dos archivos de lanzamiento particulares al directorio de instalación de Turtlebot en ROS, disponibles en el repositorio en Github del proyecto de RTAB-Map en la dirección \url{https://github.com/introlab/rtabmap_ros/}, ejecutando los siguientes comandos en una terminal:

\begin{blackcodebox}
\begin{lstlisting}[language=bash]
sudo wget -P /opt/ros/indigo/share/rtabmap_ros/launch/demo/ https://raw.githubusercontent.com/introlab/rtabmap_ros/master/launch/demo/demo_turtlebot_mapping.launch
sudo wget -P /opt/ros/indigo/share/rtabmap_ros/launch/demo/ https://raw.githubusercontent.com/introlab/rtabmap_ros/master/launch/demo/demo_turtlebot_rviz.launch
\end{lstlisting}
\end{blackcodebox}

Una vez terminadas las descargas, debemos ejecutar cada uno de estos comandos en una terminal independiente:

\begin{blackcodebox}
\begin{lstlisting}[language=bash]
roslaunch turtlebot_bringup minimal.launch
\end{lstlisting}
\end{blackcodebox}

\begin{blackcodebox}
\begin{lstlisting}[language=bash]
roslaunch rtabmap_ros demo_turtlebot_mapping.launch rgbd_odometry:=true args:="--delete_db_on_start"
\end{lstlisting}
\end{blackcodebox}

\begin{blackcodebox}
\begin{lstlisting}[language=bash]
rosrun rviz rviz -d turtlebot_navigation.rviz
\end{lstlisting}
\end{blackcodebox}

Este último comando inicia la herramienta rviz, o Visualizador (en 3D) de ROS, el cual permite obtener datos de distintos nodos y tópicos disponibles.

\begin{figure}[h]
\centering
\includegraphics[width=1.00\textwidth]{rvizmain}
\caption{Ventana principal de rviz}
\label{figure:rvizmain}
\end{figure}

Al iniciar rviz, por defecto se tendrá una base de datos nueva para comenzar a generar el mapa. En el panel izquierdo (en adelante el panel de visualizaciones), en la parte inferior, se hace clic en ``Add'' (Agregar Visualización), desplegando la ventana visible en la figura \ref{figure:rvizaddvisualization}. Desplazándose en los tipos de visualizaciones disponibles, se debe agregar \url{Map}, disponible bajo la categoría ``rviz''. Puede dejarse el nombre por defecto.

\begin{figure}[H]
\centering
\includegraphics[width=0.50\textwidth]{rvizaddvisualization}
\caption{Ventana de visualizaciones}
\label{figure:rvizaddvisualization}
\end{figure}

Una vez hecho esto, se debe desplegar el menú de opciones (con la flecha ubicada en el extremo derecho del título ``Map'') y luego seleccionar el \textit{topic} (tópico) ``\url{/rtabmap/proj_map}''.

\begin{figure}[H]
\centering
\includegraphics[width=0.40\textwidth]{rviztopic}
\caption{Selección de tópico}
\label{figure:rviztopic}
\end{figure}

La visualización mostrará la proyección horizontal del mapa, generado a medida que se recorre el entorno. Esto es visible en la figura \ref{figure:rvizmappingprojection}.

\begin{figure}[H]
\centering
\includegraphics[width=0.80\textwidth]{rvizmappingprojection}
\caption{Mapa generado (proyección horizontal)}
\label{figure:rvizmappingprojection}
\end{figure}

Adicionalmente, se puede visualizar la nube de puntos capturada, con el fin de corroborar que se esté realizando adecuadamente la generación del mapa. Para ello, se agrega una nueva visualización de la misma forma como se agregó la del mapa (Figura \ref{figure:rvizaddvisualization}), seleccionando esta vez el tipo ``MapCloud'' bajo la categoría ``rtabmap\_ros''.

\begin{figure}[H]
\centering
\includegraphics[width=0.40\textwidth]{rvizmapping}
\caption{Visualización de mapa de puntos agregada}
\label{figure:rvizmapping}
\end{figure}

Una vez agregada, se selecciona (o se corrobora que esté seleccionado) el tópico ``\url{/rtabmap/mapData}'' y se hace clic en la casilla de selección bajo el título ``Download map''.

\begin{figure}[H]
\centering
\includegraphics[width=0.40\textwidth]{rvizmappingpremap}
\caption{Obtener nube de puntos optimizada}
\label{figure:rvizmappingpremap}
\end{figure}

Rviz nos indica que reconstruirá la nube de puntos y que podría tomar un momento, tras lo cual se mostrará en la ventana principal la nube de puntos optimizada correspondiente al recorrido realizado.

\begin{figure}[H]
\centering
\includegraphics[width=0.80\textwidth]{rvizmappingmapcloud}
\caption{Nube de puntos capturada}
\label{figure:rvizmappingmapcloud}
\end{figure}

A diferencia de la versión individual de RTAB-Map, la integración con ROS guarda automáticamente los datos obtenidos (los mapas 3D y 2D) en la ruta \url{~/.ros/rtabmap.db}, por lo cual no es necesario levantar otro nodo para realizar la exportación del mapa.

Esto da por terminado el proceso y hace disponible el mapa para localización y navegación (lo cual se efectua borrando del todo el parámetro ``args'' -o eliminando el valor ``--delete\_db\_on\_start''- y pasando como nuevo parámetro ``localization:=true'' al momento de ejecutar ``demo\_turtlebot\_mapping.launch''), o si se quiere, para generar mapas de otros entornos adyacentes al ya recorrido y generado (borrando del todo el parámetro ``args'' en el paso anterior).

\paragraph{Selección de altura máxima para generación del mapa 2D}

Por defecto, el archivo de lanzamiento establece una altura de 2,0 metros como altura máxima para realizar la proyección del mapa 2D. Para cambiarlo, debe editarse el atributo ``value'' del parámetro ``proj\_max\_height'' bajo el nodo ``rtabmap'' del archivo de lanzamiento \url{/opt/ros/indigo/share/rtabmap_ros/launch/demo/demo_turtlebot_mapping.launch} como se muestra a continuación:

\begin{blackcodebox}
\begin{lstlisting}[escapechar=@, language=xml]
<node name="rtabmap" pkg="rtabmap_ros" type="rtabmap" output="screen" args="$(arg args)">
    ...
    <param name="proj_max_height" type="double" @\colorbox{electricyellow}{value=``2.0''}@ />
</node>
\end{lstlisting}
\end{blackcodebox}

Cabe destacar que el nodo no permite la selección de una altura mínima para obstáculos; lo que sí está disponible es el parámetro ``proj\_max\_ground\_angle'' (tipo de dato: doble, por defecto: 45) que es el máximo ángulo en grados entre la normal del punto a la normal del suelo, para ser etiquetado como suelo. Los puntos que tengan una mayor diferencia angular son considerados como obstáculos.

\subsection{Características técnicas del mapa generado}

Una vez que se ha realizado el recorrido parcial o completo del entorno, independientemente de si se ha obtenido el mapa 3D o la proyección 2D del mismo, obtenemos un archivo de base de datos a partir del cual pueden exportarse mapas en forma de nube de puntos, como ya vimos en la sección~\ref{sec:genenvironmentmaps}. Sin embargo, con el fin de realizar simulaciones en plataformas tales como \textit{Player/Stage}, puede ser necesario utilizar directamente la proyección 2D en forma de imagen.

En el caso particular para esta plataforma, es necesario generar un archivo \url{.world} haciendo referencia a una imagen donde los píxeles negros representan obstáculos y los blancos espacios libres (recordando la naturaleza binaria de los mapas de ocupación). El mapa que se genera tras realizar esta operación es una imagen en escala de grises en formato PGM \cite{pgmformat} (derivado del ácronimo en inglés ``\textit{Portable Grayscale Map}'' o ``Mapa Portatil en Escala de Grises'') que puede convertirse luego a formato BMP y opcionalmente transformar todos los píxeles grises de la imagen convertida a blanco.

El procedimiento exacto requiere una base de datos ya generada. Asumiendo la base de datos por defecto \url{~/.ros/rtabmap.db}:

\begin{blackcodebox}
\begin{lstlisting}[language=bash]
roscore
rosrun rtabmap_ros rtabmap _database_path:=~/.ros/rtabmap.db
rosrun map_server map_saver map:=proj_map
rosservice call /publish_map 1 1 0
\end{lstlisting}
\end{blackcodebox}

El proceso toma un tiempo en generar la proyección del mapa, pero una vez terminado, se debe obtener en la consola correspondiente a ``map\_saver'' una salida similar a la siguiente:

\begin{blackcodebox}
\begin{lstlisting}[language=bash]
[ INFO] [1441403556.668234515]: Waiting for the map
[ INFO] [1441403565.477020944]: Received a 508 X 420 map @ 0.050 m/pix
[ INFO] [1441403565.477108000]: Writing map occupancy data to map.pgm
[ INFO] [1441403565.482476945]: Writing map occupancy data to map.yaml
[ INFO] [1441403565.482683795]: Done
\end{lstlisting}
\end{blackcodebox}

Nuestro mapa ya generado se puede visualizar en el archivo ``map.pgm''. % Descripción de arquitectura y tecnologías

Capítulo 6

En este capítulo se realizará un análisis sobre los resultados de la implementación a nivel de software de la arquitectura expuesta en el capítulo anterior.

Seguridad y Autenticación

Para el proceso de autenticación, se estableció el uso de dos métodos: el método clásico de correo electrónico y contraseña, y utilizando Open Authorization (de acá en adelante, referenciado como OAuth). Como se describe en el Diagrama 2,
Diagrama 2. Funcionamiento de Open Authorization en FeathersJS
Fuente: (Clusters, 2019)


Inicio de Sesión de la Plataforma
Fuente: Propia

Para la implementación manual del proceso de registro, se hizo uso de un proceso de confirmación a través de correo electrónico, como se muestra en la imagen siguiente:
Registro de usuarios. Fuente: Propia

Formulario Inicial de Usuario. Fuente: Propia
En dicho formulario se coloca la información mínima necesario para dar inicio al usuario dentro del sistema. Al ingresar estos datos, ya se encuentra posteriormente en la Vista Inicial:

Vista inicial de la plataforma Gravedad. Fuente: Propia

En esta vista se pueden apreciar las opciones para acceder a las distintas vistas y funcionalidades.
Botón desplegable de la barra de navegación superior.
Fuente: Propia

Al acceder a la opción para ver el Perfil de Usuario (Profile), se accede a la vista básica del usuario actual.

Usuarios

Si accedemos a la Vista de Edición de Usuarios, podemos contemplar los siguientes elementos:

Primera Impresión de la Vista de Usuarios. Fuente: Propia

En ella podemos ver las seis opciones referentes a la Información General del Usuario, su Educación, Experiencia Laboral, Habilidades, Preferencias como trabajador y su Cultura Individual.

En términos de la Información General, además de la selección estándar de fotografía de perfil, género, país y biografía, se plantea al usuario la selección de otra serie de opciones asociadas a las posiciones que tuvo el usuario previamente en términos laborales, así como la experiencia que posee, los roles preferidos y los roles primarios. Estos dos últimos selectores poseen una serie de opciones precargadas que le permitirán al usuario elegir su rol particular dentro de un conjunto de opciones limitado. Así mismo, se brinda la opción al usuario de introducir información relacionada con sus redes sociales y sitio web personal.

Información Gneeral del Perfil de Usuario. Fuente: Propia

Sección de Educación en el Perfil de Usuario. Fuente: Propia

En el plano de Educación, se desarrolló una lista dinámica a la cual el usuario puede agregar su experiencia académica a través del llenado de un formulario dentro de un modal. Este modal contiene información a introducir tal como: universidad a la que asistió para la titulación, fecha de graduación, grado obtenido y título obtenido.
Formulario para añadir nueva experiencia académica. Fuente: Propia
Sección de Experiencia Laboral. Fuente: Propia


Formulario de Experiencia Laboral. Fuente: Propia

En el plano laboral, al igual que en el de educación, se hizo uso de una lista dinámica para mostrar la experiencia académica del usuario. El modal que contiene el formulario solicita que se introduzca información referente a la compañía donde trabaja (o trabajó) el usuario, el cargo que ocupó allí, el inicio y final (si aplica) de su prestación de servicios y finalmente una descripción del trabajo realizado.

Con respecto a las Habilidades Laborales del Usaurio, se permite al usuario introducir habilidades dentro de una lista precargada, así como una breve descripción de sus logros y la posibilidad de cargar el curriculum.
Especificación de Habilidades Laborales. Fuente: Propia

Respecto a las preferencias laborales, la plataforma busca entender los gustos del trabajador y de cómo se siente cómodo trabajando, tanto en su ambiente actual como en términos de salario.
Preferencias Laborales I. Fuente: Propia

Preferencias Laborales II. Fuente: Propia

Dentro de las opciones para definir las opciones para definir la cultura del usuario, se destacan una serie de preguntas asociadas a las emociones que evoca el mismo como trabajador, por ejemplo la motivación interna del usuario, la proyección a largo plazo y una parametrización en términos de la cultura de la empresa a la que desea aspirar. Esto con el objetivo de encontrar el empleo ideal para este usuario en un futuro sistema de recomendaciones.

Cultura Laboral del Usuario I. Fuente: Propia
Cultura Laboral del Usuario II. Fuente: Propia
Finalmente, el perfil de usuario tiene la siguiente forma:
Perfil de usuario. Fuente: Propia

La vista de configuración de la cuenta, por su parte, únicamente tiene información básica, tal como el nombre y apellido, dirección correo electrónico, nombre de usuario y cambio de contraseña. Las demás opciones asociadas a privacidad, preferencias de cuenta e integraciones no se encuentran disponibles.
Menu de Configuración de Cuenta. Fuente: Propia

Publicaciones

Si deseamos crear una publicación, solo hace falta hacer click sobre la entrada de texto de “Crear un nuevo Post” en la sección central del menú inicial.

Espacio de creación de Posts. Fuente: Propia

Al realizar dicha acción, se va a desplegar un modal que muestra nuevamente una entrada de texto y otra serie de campos interactivos que permiten la inserción de imágenes y archivos. Allí se puede describir el contenido textual de la publicación, así como quién está realizando la publicación (si es una startup, un producto o un usuario) y las imágenes y contenido multimedia asociado.
Formulario para la creación de Publicaciones. Fuente: Propia

Al crearse una publicación, la misma es listada en tiempo real en la vista principal. La publicación también puede ser editada o eliminada dentro del menú en la parte superior de la publicación. Todas las publicaciones, al ser creadas, incluyen las siguientes funcionalidades:
La función de interacción “Me gusta”, para indicar apoyo a una publicación.
La función de comentar, que permite a los usuarios seguidores de otro usuario, startup o producto, emitir opiniones. También pueden editar o eliminar dichos comenarios.
La función de compartir publicación, para que dicha publicación sea listada también para los usuarios seguidores de cualquiera que comparta la publicación y no únicamente de quien la creó.
Publicación listada en la Vista Inicial. Fuente: Propia

Si el usuario clickea sobre la fecha y hora de la publicación, podrá ingresar a una vista dedicada donde únicamente tendrá la publicación presente. Esta vista dedicada puede ser compartida a través de un link, por si se desea llevar a un lugar externo a la plataforma. Todo usuario puede crear nuevos perfiles, tanto para una Startup como para un Producto.  Asímismo, se puede solicitar la posibilidad de ser inversor o de convertirse en un Aliado para que, manualmente, un desarrollador de Ignis Gravitas incorpore a dicho individuo al listado de inversores o aliados claves previa aprobación de la compañía Ignis Gravitas, Inc.

Menú desplegable para la creación de perfiles. Fuente: Propia

Startups	

Al seleccionar Crear Perfil de Startup, dicho usuario accederá a una vista de creación particular que se enfoca únicamente en la vista de las startups. Esta vista contiene una serie de secciones: Resumen, Personas, Cultura, Fondos y Trabajos.

Formulario de Creación de Startups. Fuente: Propia

Cada una de estas secciones contiene dentro de sí otra serie de datos de entrada. En la primera sección, se solicita la información básica de la startup, que se engloba en: Nombre de la Startup, dirección de correo electrónico, descripción básica de la startup, discurso de venta de la startup (pitch), país de origen, ciudad de origen, dirección física, dirección del sitio web de la startup, enlaces a las Redes Sociales de la startup, selección de los mercados a los cuales apunta la startup, enlace a un vídeo introductorio de la startup y, finalmente, historias de usuarios satisfechos con el trabajo de la startup.
Formulario de Información Básica de la Startup. Fuente: Propia

En la segunda vista, se pueden realizar las adiciones de membresías y hacer una descripción somera de lo que significa el equipo para la startup.

Formulario de la Sección Personas. Fuente: Propia

Al hacer click en el botón de agregar, se despliega un nuevo formulario que permite invitar nuevos miembros a la Startup. Estos nuevos miembros aparecerán luego en el perfil de la startup.

Formulario para añadir un nuevo miembro. Fuente: Propia

En la sección cultura, se añaden nuevas entradas de texto referentes a la descripción cultural de la startup, así como una serie de entrada de archivos para subir imágenes que muestren referencias a la compañía, como sus sitios de trabajo, personal, productos, etc. Además, se pueden agregar los beneficios y ventajas que ofrece la startup a cada uno de sus empleados.
Formulario de la Sección Cultura. Fuente: Propia

Añadir un nuevo beneficio. Fuente: Propia

En la sección de Financiamiento, se puede agregar la cantidad de dinero recibida hasta el momento, así como los inversores que han hecho contribuciones a la startup.

Entrada de datos de la sección Financiamiento. Fuente: Propia


Formulario para añadir un inversor. Fuente: Propia

En la sección de trabajos, se puede agregar la descripción general de la startup, así como las oportunidades de trabajo asociadas a la misma.

Entrada de datos en la sección Trabajo. Fuente: Propia
Formulario para propuestas de trabajo. Fuente: Propia
Productos

Al buscar crear un producto, se muestran el siguiente formulario:
El mismo, dividido en tres pestañas (“Resumen”, “Presentación” y “Personas”) permite detallar el producto ofertado. Los productos son, en una primera versión, independientes de las startups.
Sección Presentación. Fuente: Propia
Edición de Productos. Fuente: Propia
Menú General
Desde la Vista Principal, se pueden acceder a las distintas opciones de la plataforma:
Pila de Opciones de la Plataforma Gravitas. Fuente: Propia


Trabajos

El listado de trabajos es accesible desde la Pila de Opciones en el menú lateral. La misma posee la siguiente interfaz gráfica:

Vista de Listado de Trabajos

Del lado izquierdo, se ven las distintas ofertas de trabajo a las cuales puede aplicar un desarrollador, y a la derecha, una descripción detallada de la misma. Al hacer click sobre “Aplicar”, se muestra el siguiente formulario:
Formulario de aplicación a propuestas de trabajo. Fuente: Propia

Listado de Startups y Productos

Listado de Startup. Fuente: Propia

Al acceder al listado de startups, se puede contemplar cada una de las startups creadas por los distintos usuarios, así como aquellas sugeridas durante la selección del empleo previo, como un perfil creado por la comunidad y que no puede ser gestionado hasta no haber sido pedido y su autenticidad verificada.

Cada perfil contiene la información introducida en un principio de manera ordenada, así como los posts que ha publicado previamente:

Perfil de una startup. Fuente: Propia

En el listado de productos, por su parte, se ven los productos creados junto a su imagen, su nombre, su descripción básica, así como las etiquetas que lo describen. Así mismo, se tiene la posibilidad de votar al producto, para posicionarlo en la parte superior de la lista para los usuarios.
Lista de Productos. Fuente: Propia
Al acceder al perfil del producto, se puede ver toda la información introducida inicialmente para describir al mismo
Perfil de producto. Fuente: Propia

El resto de las vistas (“Invitaciones”, “Inversores”, “Mentores”, “Asesores” y “Aliados”) al momento de finalizar este proyecto no habían sido implementadas por lo que únicamente se replicó el mismo formato que en el listado de startups.

 % Descripcion de la plataforma funcional

\chapter{Conclusión y Recomendaciones}

\section{Conclusión}

Las Redes Sociales como plataformas tecnológicas y la necesidad de la creación de valor han dilucidado, como se ha podido contemplar durante el desarrollo de este proyecto, ser un potencial de actividad económica gigantesco, que puede impulsar cambios masivos en la calidad de vida de las personas, al emparejar emprendedores con inversores, así como empleados con empresas propias.

Asimismo, el desarrollo de plataformas haciendo uso de marcos de trabajo de desarrollo web de este nivel de complejidad exigen de un entendimiento muy claro de los requerimientos del producto, razón por la cual una gran parte del tiempo de desarrollo fue usado para este fin, a modo de detallar de la manera máx explícita posible cada uno de los requerimientos necesitados.

\section{Recomendaciones}

El software descrito en este documento posee la capacidad de fungir como base de múltiples desarrollos a nivel de software que permitiesen facilitar la creación de redes sociales. Para ello, se pueden seguir las siguientes recomendaciones:

\begin{itemize}
	\itemsep1pt \parskip1pt \parsep1pt
	\item Continuar con el desarrollo del proyecto en las siguientes funcionalidades: desarrollo de un sistema de Autenticación a dos pasos, optimización de tiempos de carga, incorporación de un chat entre usuarios, desarrollo de preferencias de usuario y soporte a integración con otras plataformas.

	\item Publicar la base de código como un paquete de acceso público o privado de NPM.

	\item Proponer un proyecto de marketing digital para hacer crecer la presencia de la plataforma Ignis Gravitas en la red.
\end{itemize}

\clearpage
\appendix
\renewcommand{\appendixname}{Apéndice}
\renewcommand{\appendixtocname}{Apéndices}
\renewcommand{\appendixpagename}{Apéndices}
\appendixpage
\noappendicestocpagenum
\addappheadtotoc
\chapter{Ignis Gravitas}

\section{Descripcion de Ignis Gravitas} \label{App:DescripcionLasdai}

Ignis Gravitas, Inc. es una startup registrada en Delaware, Estados Unidos. Su objetivo es expandir la cultura empresarial y el crecimiento de las startups en todo el mundo.

\subsection{Personal}

\subsubsection{Miembros}

\begin{itemize}
    \itemsep1pt \parskip0pt \parsep0pt
    \item Dr.\ Gerard Páez Monzón. (CEO)
    \item Br.\ Alfredo Ferreira Pace
    \item Br.\ Diego Benitez Peña
    \item Br.\ Aries Lugo
\end{itemize}

\subsection{Mision}

La misión de Ignis Gravitas, Inc. es la vinculación con el mundo emprendedor a través de la plataforma tecnológica y el segundo es la estructuración de una propuesta de valor complementaria como es la Academia Ignis Gravitas, donde se forman todos los interesados en el emprendimiento, con el objetivo de promover una nueva alternativa en la educación, adaptada a las nuevas demandas del mundo emprendedor.

\subsection{Vision}

Desde el punto de vista logístico y operativo, Ignis Gravitas tiene una personalidad enfocada a la aceleración de las diferentes Startups que surgen en su plataforma. Estos servicios se complementarán próximamente con los servicios de la Academia Ignis Gravitas, con el objetivo de aportar una visión integral en el desarrollo del proceso emprendedor al que se enfrentan las Startups en su día a día.


% Estilo de la bibliografía
\bibliographystyle{apa}
% Modificada ya que la original está en inglés.

% ***************************************************************** %
% Para agregar toda la bibliografia del archivo .bib
% solo descomente el siguiente comando
% ***************************************************************** %
\nocite{*}

% ***************************************************************** %
% Nombre del archivo con extensión .bib en donde se almacena la bibliografía
\bibliography{Referencias}

% ***************************************************************** %
% FIN DE
% Cuerpo
% ***************************************************************** %

\end{document}
