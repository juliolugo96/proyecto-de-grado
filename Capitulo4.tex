\chapter{Extracción de Requerimientos}

En este capítulo se aglutinan los requerimientos para la construcción posterior de la plataforma, así como el formato utilizado para los mismos.

\section{Formato de Requisitos}

Los requerimientos funcionales y no funcionales de la aplicación fueron extraídos de la aplicación durante reuniones semanales continuas con el CEO de Ignis Gravitas, Inc. Los requerimientos funcionales fueron extraídos haciendo uso del siguiente formato, especificado por Holmes (2018), donde haciendo uso de las historias de usuario clásicas de las metodologías ágiles, construyó una estructura para las mismas donde se unificaban aspectos para la satisfacción de los clientes como elementos propios de la ingeniería \cite{edx}. En esta estructura, se enumeran pues, cinco elementos principales:


\begin{description}
    \item[Rol-Meta-Beneficio:] Es la exposición realizada por el cliente para definir quién va a beneficiarse de una funcionalidad en específico, para qué se desea construir y por qué motivación personal se quiere lograr su construcción.
    \item[Limitaciones:] Describe las dependencias existentes entre el requerimiento que está siendo descrito y otros que deben ser encontrarse en estado funcional antes de él.
    \item[Definición de realizado (condiciones de satisfacción):] Son las condiciones que pone a prueba el interesado para afirmar que la característica fue terminada exitosamente.
    \item[Tareas de ingeniería:] Información que es de importancia para los desarrolladores al momento de construir la característica.
    \item[Estimación de esfuerzo:] Representa en términos de unidades de trabajo, cuánto será necesario invertir en desarrollar la característica descrita.
\end{description}

A efectos de este proyecto, las limitaciones son expresadas en la matriz de requerimientos. A cada historia de usuario se le ha añadido la prioridad siguiendo el esquema descrito MoSCoW. Se buscó seguir principios INVEST para el desarrollo de las historias de usuario.

\section{Formato de Priorización de Requisitos}

\begin{description}
    \item[M (Must have) Debe tener:] Requisito que tiene que estar implementado en la versión final del producto para que la misma pueda ser considerada un éxito.
    \item[S (Should have) Debería tener:] Requisito de alta prioridad que en la medida de lo posible debería ser incluido en la solución final, pero que llegado el momento y si fuera necesario, podría ser prescindible si hubiera alguna causa que lo justificara.
    \item[C (Could have) Podría tener:] Requisito deseable pero no necesario, se implementaría si hubiera posibilidades presupuestarias y temporales.
    \item[W (Won’t have) No tendrá esta vez:] Hace referencia a requisitos que están descartados de momento pero que en un futuro podrían ser tenidos en cuenta y ser reclasificados en una de las categorías anteriores.
\end{description}

\section{Listado de Requerimientos}

A continuación, se listan los requerimientos extraídos de las reuniones con el CEO de Ignis Gravitas, Inc.

\subsection{Registro de Usuario Manual}

\begin{description}
    \item[Prioridad:] M
    \item[Rol-Meta-Beneficio:] Como USUARIO no registrado de Gravitas necesito poder crear un usuario en el sistema para tener acceso a las funcionalidades de la plataforma.
    \item[Limitaciones:] El usuario no debe poseer ninguna información de registro previo en la plataforma.
    \item[Condiciones de Satisfacción:]
        \begin{enumerate}
            \item Solo se le solicitará al usuario introducir su nombre, apellido, correo, clave y aceptación de los términos y condiciones de uso para iniciar el proceso de registro.
            \item Luego de ingresados los datos, el sistema debe enviar un correo a la dirección ingresada por el usuario.
            \item Una vez verificada la cuenta de correo electrónico, se le solicitará al usuario introducir sus preferencias, esto se utilizará posteriormente para filtrar contenido personalizado.
            \item Se solicitará el resto de la información requerida según el tipo de perfil elegido y los planes de pago para finalizar la creación del usuario.
            \item Una vez finalizado el registro se enviará a la vista inicial.
        \end{enumerate}
    \item[Tareas de Ingeniería:]
        \begin{enumerate}
            \item Desarrollar la vista de Registro inicial para correo y clave.
            \item Desarrollar email de validación de correo.
            \item Desarrollar la vista de Continuación de Registro, donde se mostrarán todas las posibles preferencias, incluyendo el tipo de perfil a desarrollar.
            \item Desarrolla las vistas de formulario para cada tipo de perfil.
        \end{enumerate}
    \item[Unidades de Trabajo:] 160h
    \item[Dependencias:] Ninguna.
\end{description}

\newpage


2) Verificación de Correo Electrónico

	Prioridad M

	1.-	Rol-Meta-Beneficio: Como USUARIO no registrado de Gravitas, necesito verificar mi correo electrónico para poder proseguir mi proceso de Registro.
	2.- 	Limitaciones: Se debe introducir una dirección de correo automatizada.
	3.-	Condiciones de Satisfacción:
		3.1.- Al ingresar mis datos básicos en el sistema Gravitas, el sistema debe enviar un correo electrónico a la dirección suministrada.
		3.2.- El correo electrónico debe contener un texto base explicando el proceso de registro y debe incluir un enlace a un formulario de preferencias de usuario.
		3.3.- Al ingresar al enlace, la cuenta de correo debe ser automáticamente verificada automáticamente en el sistema.
		3.4.- El enlace debe ser válido únicamente por 24 horas.
		3.5.- Al ingresar a un enlace no validado, se debe redirigir a la vista de registro e incluir un mensaje de error.
		3.6.- Al invalidarse un enlace, debe también eliminarse cualquier información persistida del registro inicial del usuario.

	4.- 	Tareas de Ingeniería
		4.1.- Implementar un remitente programado (mailer) para el envío de  correos a través del protocolo SMTP.
		4.2.- Desarrollar la vista para introducción de datos de usuario.
	5.-	Unidades de Trabajo: 16h
	6.-	Dependencia: 1


3) Registro de Usuario mediante Open Authorization (OAuth)

	Prioridad M

	1.-	Rol-Meta-Beneficio: Como USUARIO no registrado de Gravitas deseo crear mi usuario dentro de la plataforma utilizando mis credenciales ya verificadas de servicios conocidos, para así agilizar mi acceso a la plataforma como usuario verificado.
	2.-	Limitaciones: El usuario no debe haber iniciado sesión previamente ni tener una cuenta previamente asociada al correo de Gmaiail.
	3.-	Condiciones de Satisfacción:
		3.1.- El sistema debe brindar las opciones de Google, Facebook, LinkedIn y Github para realizar el registro de un nuevo usuario.
		3.2.-	El usuario debe elegir su proveedor preferido para realizar la autenticación.
		3.3.-	El sistema debe desplegar la interfaz de autenticación correspondiente al proveedor elegido.
		3.4.-	Al terminar el proceso de validación con el proveedor, el sistema debe almacenar los datos básicos del usuario (Nombre, apellido, correo electrónico) dentro de la información básica personal.
		3.5.-	El sistema debe redirigir a la vista de preferencias al usuario.
		3.6.-	Si el proveedor presenta o envía un error, informar al usuario con un “Fallo en la autenticación del servicio X” donde X es el servicio elegido.
	4.- Tareas de Ingeniería
		4.1.- Generar la clave de acceso a la API de O-Auth en la cónsola de desarrollo de Google.
		4.2.- Entender cómo funciona el sistema de Autenticación mediante Oauth de FeathersJS-
		4.3.- Configurar la creación de un nuevo usuario con los datos proveídos por Google.
	5.- Unidades de Trabajo: 8h.
	6.- Dependencias: 1


4) Inicio de Sesión Manual

	Prioridad M

	1.-	Rol-Meta-Beneficio: Como USUARIO de Gravitas necesito poder iniciar sesión en el sistema para manejar las herramientas ofrecidas por la plataforma.
	2.-	Limitaciones:   Se debe permitir heredar la sesión de Ignis en el sistema Gravitas; además, ya el usuario debió registrarse con el correo asociado al servicio de autenticación.
	3.-	Condiciones de Satisfacción:
		3.1.-	Solicitar usuario y contraseña para autenticar al usuario.
		3.2.-	Validación de la información en servidores para poder continuar.
		3.3.-	Si el usuario no ha finalizado el proceso de registro, debe ir a finalizarlo para continuar.
		3.4.-	Si el usuario finalizó el proceso de registro y la información suministrada es correcta, debe dirigir a la vista inicial.
		3.5.-	Si la información suministrada no coincide con la persistida en el sistema, el mismo debe emitir un error de “Correo o Contraseña inválido”.
	4.-	Tareas de Ingeniería:
		4.1.-	Desarrollar la vista de inicio de sesión, solicitando correo y clave para continuar, una vez autenticado, se debe evaluar si este usuario debe finalizar el proceso de registros o si puede acceder a la vista incial.
	4.2.- Estudiar los patrones de desarrollo de Single Sign On (SSO) entre una pltaforma de NodeJS y Ruby On Rails.
	4.3.- Implementar un Single Sign On (SSO) para persistir las sesiones entre ambas plataformas.
	5.-	Unidades de Trabajo: 8h.
	6.- 	Dependencia: 1. 2.
5) Inicio de Sesión mediante Open Authorization (OAuth)
	
Prioridad M

	1.-	Rol-Meta-Beneficio: Como USUARIO de Gravitas deseo acceder a mi usuario dentro de la plataforma utilizando mis credenciales ya verificadas de servicios conocidos, para así agilizar mi acceso a la plataforma como usuario verificado.
	2.-	Limitaciones: Se debe permitir heredar la sesión de Ignis en el sistema Gravitas; además, ya el usuario debió registrarse con el correo asociado al servicio de autenticación.
	3.-	Condiciones de Satisfacción:
		3.1.- El sistema debe brindar la opción de Google para permitir el inicio de sesión de un usuario.
		3.2.-	El sistema debe desplegar la interfaz de autenticación correspondiente al proveedor elegido.
		3.3.-	Al terminar el proceso de validación con el proveedor, el sistema debe redirigir a la pantalla inicial.
		3.4.-	Si el proveedor presenta o envía un error, informar al usuario con un “Fallo en la autenticación del servicio X” donde X es el servicio elegido.
	4.-	Unidades de Trabajo: 8h.
	5.-	Dependencias: 1, 2.

6) Autenticación en dos pasos

	Prioridad W

	1.-	Rol-Meta-Beneficio: Como USUARIO de Gravitas necesito garantizar la seguridad de mis datos con una autenticación personal, rápido y confiable para las operaciones  para la plataforma.
	2.-	Limitaciones: Se debe permitir heredar la sesión de Ignis en el sistema Gravitas.
	3.-	Condiciones de Satisfacción:
		3.1.- El modo de Autenticación en dos pasos solo estará activo si el usuario así lo seleccionó en sus preferencias.
		3.2.- Al momento de realizar una operación de inicio de sesión y/o intercambio de valor, el sistema debe desplegar la interfaz de autenticación en dos pasos.
		3.3.-	El sistema debe solicitar al usuario un código generado automáticamente.
		3.4.-	El sistema de correos debe enviar un correo electrónico cuyo contenido sea el código generado.
		3.5.-	Al introducirse el código correctamente, el sistema debe redirigir según corresponda a la vista de confirmación del intercambio de valor ó  a la vista de inicio de sesión.
		3.6.-	De introducirse el código erróneamente, el sistema debe indicar un mensaje de error con el mensaje “Código Inválido”.

	4.-	Tareas de Ingeniería:
		4.1.- Integrar un sistema generador de códigos de confirmación.
		4.2.-	Reciclar el JWT KEY de Ignis en Gravitas.
		4.3.-	Desarrollar algún tipo de funcionalidad que permita encapsular el proceso de autentificación para ser usada en todas los endpoints que lo requieran.
		4.4.-	Configurar el backend para permitir el envío de correos electrónicos.
		4.5.-	Desarrollar funcionalidad que soporte el envío de todo tipo de correos electronicos.
		4.6.-	Desarrollo del modulo de envío de correos para autenticación en dos pasos.
	5.-	Unidades de Trabajo: 16h.
	6.- 	Dependencia: 4, 5.


7) Planes de pago por el servicio

	Prioridad W

	1.-	Rol-Meta-Beneficio:	Como INTERESADO de la plataforma Gravitas, quiero que los USUARIOS paguen por el uso del servicio ofrecido para MANTENER económicamente viable el proceso de crecimiento y desarrollo de la plataforma misma.
	2.-	Limitaciones: El sistema debe manejar la posibilidad de que por razones de negocios, los planes sean temporalmente deshabilitados.
	3.-	Condiciones de Satisfacción:
		3.1.-	El sistema debe indicarle al usuario que no hay ciertas funcionalidades habilitadas hasta no realice el pago correspondiente a su plan.
		3.2.- El sistema debe ofrecerle un enlace a un procesador de pagos.
		3.3.-	El usuario debe tener una versión corta de los Términos y Condiciones del uso de la plataforma antes de iniciar el proceso de pago, así como un enlace al documento completo.
		3.7.-	El usuario debe ser llevado a la vista de personalización de la experiencia de usuario luego del pago de los planes.
	4.-	Tareas de Ingeniería:
		4.1.-	Investigar la integración de Stripe con el framework a utilizar.
		4.2.-	Mantener comunicación constante con los interesados. Se recomienda una metodología de Programación Extrema de fácil adaptación a los cambios en los requerimientos iniciales.
		4.4.-	Al terminar el proceso de pagos, la información del pago debe ser registrada en la base de datos de la plataforma.
		4.3.-	La información del pago debe estar disponible como un registro tanto para el usuario como para los administradores del sitio.
	5.-	Unidades de Trabajo: 160h.
	6.- 	Dependencia: 1, 2, 3.


8) Interacción inicial y de configuración básica

	Prioridad C

	1.-	Rol-Meta-Beneficio: 	Como USUARIO de la plataforma Gravitas, me gustaría poder ajustar mis preferencias básicas para obtener mejores recomendaciones de publicaciones y una mejor experiencia de usuario general.
	2.-	Limitaciones: El usuario debe estar registrado y/o haber realizado sus operaciones de pago.
	3.-	Condiciones de Satisfacción:
		3.1.-	El usuario debe elegir su rol en la plataforma según los que se encuentran disponibles: “trabajador”, “mentor”, “inversionista”, “emprendedor”.
		3.2.-	El usuario debe elegir una serie de intereses personales (utilizados para configurar los posts recomendados de su inicio).
		3.2.-	Si el usuario eligió la opción “emprendedor”, se debe brindar la opción de crear un emprendimiento.
			3.2.1.-	El usuario debe llenar un formulario con información básica; nombre, logotipo, descripción y área de emprendimiento.
			3.2.2.-	El usuario debe responder si es un emprendedor con experiencia o un novato en busca de aprender.
				3.2.2.1.-	Si elige la primera opción, pasará inmediatamente a una vista donde enviará las invitaciones a los miembros de su equipo y sucesivamente será enviado a la vista inicial.
				3.2.2.2.-	Si elige la segunda, debe mostrarse una interfaz de tutorial donde se muestren relatos gráficos, y vídeos del uso de las herramientas de la plataforma Gravitas, con la opción de saltarlos y proseguir con el flujo de la primera opción.
		3.3.-	Si el usuario eligió la opción “trabajador” o “mentor”, debe pasar luego a llenar un formulario donde indique información laboral básica, como experiencia y educación.
			3.3.1.- El usuario debe poder subir su CV en formato PDF.
			3.3.2.-	Luego de introducida esta información, si dicho usuario ya posee una invitación a una startup, deberá aparecer la opción de aceptar o rechazar la invitación, de lo contrario, será redirigido a la página de inicio.
		3.4.-	Si el usuario eligió la opción “inversionista”, el mismo debe introducir los nombres de los emprendimientos dentro de los cuáles ha invertido capital.
			3.4.1.- El usuario debe indicar en un posterior campo un estimado total de lo invertido en ellos. Terminado ello, pasa a ser redirigido a la página de inicio.
	4.-	Tareas de Ingeniería:
		4.1.- Deben construirse las vistas para la entrada de datos, así como la información correspondiente.
		4.2.- Se debe negociar con el interesado para definir si el manejo de roles y si se puede replantear dichos roles a únicamente “Administrador” y “Usuario Común”, con el objetivo de dejar los roles solo a las membresías de startups.
	5.-	Unidades de Trabajo: 32h.
	6.-	Dependencias: 4, 5.

9) Vista tutorial e invitaciones automáticas

	Prioridad W

	1.-	Rol-Meta-Beneficio: Como USUARIO de la plataforma Gravitas, deseo recibir información instruccional del sitio para poder aprovechar de manera óptima todas sus herramientas.
	2.-	Limitaciones: El usuario debe haber iniciado sesión mediante alguno de los métodos disponibles.
	3.-	Condiciones de Satisfacción:
		3.1.- Al usuario seleccionar que desea aprender a emprender, deben desplegarse una interfaz gráfica que contenga un listado de los tópicos más importantes del emprendimiento.  Inicialmente, serán cuatro puntos que estarán cada uno de ellos relacionados con las casillas del modelo de negocios: la propuesta de valor (englobando idea y construcción del producto), infraestructura (englobando las actividades, aliados y recursos claves), clientes (relacionado con las relaciones, canales y segmentos) y finalmente los flujos monetarios (estructuras de costos y flujos de ingresos).
		3.2.- Al hacer click sobre cada uno de ellos, este debe incluir un vídeo, una serie de gráficos y un texto explicativo con una fuente agradable para la lectura y un tiempo estimado de lectura.
		3.3.-	Cuando el usuario finalice la lectura o el vídeo, debe indicarse que ese tutorial particular fue completado y ofrecer un enlace al tutorial siguiente.
		3.4.-	Al completarse todos los tutoriales, el usuario pasará a invitar a todos los usuarios que crea competentes en su emprendimiento, con su rol correspondiente y su responsabilidad en el mismo. Las responsabilidades son: CEO (Chief Executive Officer), CTO (Chief Technology Officer), CMO (Chief Marketing Officer), CFO (Chief Financial Officer), CIO (Chief Information Officer), CCO (Chief Communications Officer). Finalizado esto, debe dirigirse a la vista inicial.
	4.-	Tareas de Ingeniería:
		4.1.- Mantener comunicación constante con el equipo de diseño gráfico para integrar mejoras y cambios a la experiencia de usuario (UX) y a los gráficos.
		4.2.- Construir una interfaz de usuario para la disposición de iFrames  con contenido exclusivo de Ignis Gravitas.
	5.-	Unidades de Trabajo:  160h.
	6.-	Dependencias: 4, 5.

10) Vista Inicial

	Prioridad M

	1.-	Rol-Meta-Beneficio: Como USUARIO de la plataforma Gravitas, deseo poder interactuar con las publicaciones más recientes y de mayor interés de mis contactos para tener información que pueda resultar de utilidad para mí y para el emprendimiento al que pertenezco.
	2.-	Limitaciones: El usuario debe estar registrado, haber iniciado sesión y haber definido sus preferencias de usuario.
	3.-	Condiciones de Satisfacción:
		3.1.-	 Al culminar el proceso de configuración del emprendimiento, debe mostrarse una interfaz básica que muestre a los usuarios publicaciones basadas en sus preferencias y/o al emprendimiento al que pertenecen de manera inmediata, así como a los demás usuarios y emprendimientos que sigue dentro de la plataforma.
		3.2.-	La vista inicial debe mostrar cuatro secciones visualmente separadas:
			3.2.1.- La vista de publicaciones (con una información básica inicial la primera vez que se inicia sesión, en el centro)
				3.2.1.1.-	Cada publicación debe contener el nombre del emprendimiento o el usuario que la publica, su foto, fecha y hora de publicación, imagen de la publicación (si la posee) y una descripción asociada.
				3.2.1.2.-	La publicación debe indicar la fuente de su contenido si el mismo fue extraído desde un sitio externo.
				3.2.1.3.-	Cada publicación se le debe poder dar “Me Gusta”,  agregar un comentario (donde se puedan hacer menciones) y compartirse como una publicación propia (de no serlo).
				3.2.1.4.-	El usuario debe ser capaz de crear sus propias publicaciones.
				3.2.1.5.-	El usuario debe poder anexar imágenes, enlaces a sitios externos, etiquetar a otras personas en una publicación.
			3.2.2.-	Propaganda alusiva al producto Ignis, así como sugerencias de contactos o emprendimientos a seguir (derecha).
				3.2.2.1.- Debe mostrar enlaces a secciones básicas como el “Perfil del Emprendimiento” y “Configuraciones”.
			3.2.3.-	Accesos directos a las herramientas de Gravitas e información básica del perfil social (izquierda):
				3.2.3.1.- Debe mostrar el “Número de seguidores” del usuario.
			3.2.4.-	Aaccesos directos al perfil personal y a una barra de búsqueda (barra de navegación superior) e íconos para un chat (inferior).
	4.-	Tareas de Ingeniería:
		4.1.-	Definir el formato de publicaciones y definir quienes pueden realizarlas.
		4.2.- Desarrollar un patrón reactivo mediante web sockets para que, al momento de crearse nuevas publicaciones en la capa de datos, la interfaz reciba en tiempo real esta nueva información y mande la información al usuario.
		4.3.- Implementar un sistema de cargar perezosa (lazy loading) para poder cargar de manera asíncrona las distintas publicaciones que estén en la base de datos.
	5.-	Unidades de Trabajo: 160h
	6.-	Dependencias: 5.


















\section{Definición}

\textit{Robot Operating System} (Sistema Operativo de/para Robot) o sencillamente ROS es, tal como su nombre implica, un sistema operativo para robots, de forma similar a los sistemas operativos para computadores de escritorio o servidores. Desarrollado y mantenido por la empresa Willow Garage desde 2008 hasta 2013, siendo tomada su dirección en ese año por la Fundación de Robótica de Código Abierto, es una colección de herramientas, bibliotecas y convenciones que buscan simplificar la tarea de crear comportamientos de robot robustos y complejos a lo largo de una amplia variedad de plataformas robóticas.

La justificación de porqué hacer esto es porque decididamente, crear software robótico de propósito general y verdaderamente robusto es difícil, ya que si bien para un ser humano algunos problemas son triviales, no lo son en lo absoluto al momento de tomar en cuenta las grandes variaciones entre instancias de tareas y entornos. Lidiar con estas variaciones es tan complicado que ningún individuo, laboratorio o institución pudiera esperar llevarlo a cabo por su propia cuenta.

Por ello, ROS fue construido desde cero con el fin de alentar el desarrollo de software robótico de forma colaborativa. Un ejemplo de esto es que, un laboratorio podría tener expertos en cartografía o mapeado de interiores y podría contribuir un sistema de excelente calidad para la producción de mapas. Otro grupo podría tener expertos en el uso de mapas para navegar, y otro grupo podría haber descubierto un enfoque de visión por computador que funciona bien para el reconocimiento de objetos pequeños entre el desorden. ROS fue diseñado específicamente para grupos como éstos para colaborar y construir sobre el trabajo del otro. \cite{aboutros}

Además, ROS es Software Libre y está distribuido bajo la licencia BSD, permitiendo el desarrollo de proyectos comerciales y no-comerciales. Una característica importante en cuanto a la arquitectura (que se detallará más adelante) es que ROS funciona a través de comunicación entre procesos, sin requerir que los módulos sean enlazados dentro del mismo ejecutable, por lo que cualquier sistema construido usando ROS como base puede tener control detallado sobre las licencias de software que utilicen sus módulos, ya sean GPL, BSD o cualquier otra hasta propietaria. \cite{quigley2009ros}

\section{Características Principales}

Se pueden comentar las siguientes:

\begin{description}
	\item[Comunicación entre pares:] los sistemas robóticos complejos con múltiples enlaces podrían tener varios computadores de a bordo (para realizar tareas paralelas) conectados a través de una red. La ejecución de un maestro central podría dar lugar a la congestión severa en un enlace determinado. Usando una comunicación peer-to-peer o entre pares evitaría este problema. En ROS, una arquitectura peer-to-peer acoplado a un sistema de memoria intermedia o \textit{buffer} y un sistema de búsqueda (un servicio de nombres llamado ``maestro'' en ROS), le permite a cada componente dialogar directamente con cualquier otro, de forma sincrónica o asincrónica como sea necesario.

	\item[Gratuito y de código abierto:] Ser una plataforma de código abierto ofrece la reutilización de funciones ya existentes proporcionadas por muchos otros usuarios de ROS. Su código se suministra en repositorios como \textit{stacks}, o ``pilas''. Otras personas han desarrollado capacidades sorprendentes para los robots que han sido ``de código abierto'' y son relativamente fáciles de añadir de forma incremental utilizando el entorno de desarrollo de ROS.

	\item[Delgado:] Para combatir el desarrollo de algoritmos que se ``enredan'' o vinculan en un grado mayor o menor con el sistema operativo del robot y, por tanto, son difíciles de reutilizar posteriormente, los desarrolladores de ROS han planificado que los controladores y otros algoritmos sean contenidos en ejecutables independientes. Esto garantiza la máxima reutilización y, sobre todo, mantiene reducido su tamaño. Este método hace que ROS sea fácil de usar y ubica la complejidad en las bibliotecas. Esta disposición también facilita las pruebas unitarias y los sistemas desarrollados puede ser completamente independientes de otro sistema.

	\item[Multi-lenguaje:] ROS es independiente del lenguaje, y se puede programar en varios lenguajes. La especificación ROS trabaja en la capa de mensajería. Las conexiones \textit{peer-to-peer} se negocian en XML-RPC, que existe en un gran número de lenguajes. Para soportar un nuevo lenguaje, se pueden reenvolver clases C ++ (lo cual se hizo para el cliente Octave, por ejemplo) o se escriben clases permitiendo que se generen mensajes. Estos mensajes se describen en IDL (\textit{Interface Definition Language}). \cite{quigley2009ros}
\end{description}

\section{Arquitectura}

ROS está implementado bajo los siguientes conceptos fundamentales:

\begin{itemize}
	\itemsep1pt \parskip1pt \parsep1pt
	\item Nodos: Son procesos que realizan cálculos; en el contexto de ROS, este término es intercambiable con ``módulo de software'' ya que está diseñado para ser altamente modular: un sistema está compuesto típicamente de muchos nodos.

	\item Mensajes:	Los nodos se comunican entre si al pasar mensajes, que no es más que una estructura de datos de tipo estricto. Los tipos de datos soportados pueden ser estándar (entero, flotante, booleano, etc.), así como también arreglos de estos o constantes. Un mensaje puede estar compuesto por varios mensajes y el nivel de anidamiento al que pueden llegar es arbitrario.

	\item Tópicos: Un nodo publica un mensaje a través de un tópico, que es sencillamente una cadena de caracteres tal como ``odometría'' o ``mapa''. Un nodo que esté interesado en un tipo de dato específico se suscribirá al tópico apropiado. En cualquier momento dado, pueden existir múltiples publicadores o suscriptores de forma concurrente para un tópico particular y un nodo puede publicar o suscribirse a múltiples tópicos. Por lo general, los publicadores y suscriptores no están al tanto de la existencia del otro.

	\item Servicios: Si bien el modelo publicar-suscribir basado en tópicos es un paradigma de comunicaciones flexible, el esquema de enrutamiento de ``emisión'' no es apropiado para las transacciones síncronas, lo cual puede simplificar el diseño de algunos nodos. A esto se le llama ``servicio'' en ROS, definidos por un nombre y un par de mensajes tipados, uno para la petición y otro para la respuesta. Es de notar que, a diferencia de los tópicos, solo un nodo puede anunciar un servicio con un nombre particular; por ejemplo, solamente puede haber un servicio llamado ``clasificar\_imagen''. \cite{quigley2009ros}
\end{itemize}

\section{Requisitos de Instalación}

ROS está organizado en distribuciones, que cuentan cada una con sus respectivos requisitos de instalación. Por consenso, cada mes de mayo se lanza una nueva distribución de ROS, y las distribuciones de años pares son de soporte extendido, con 5 años de soporte. Las distribuciones de años impares son distribuciones regulares, con soporte por 2 años. Igualmente, la plataforma soportada por defecto es Ubuntu Linux, mientras que otros sistemas operativos, tales como OS X, Android, Arch Linux, Debian y Windows están bajo soporte experimental, por parte de la comunidad.

Al momento de la elaboración de este proyecto, la versión instalada es ``\textit{Indigo Igloo}'' (con soporte LTS o \textit{Long Term Support} -- Soporte de Larga Duración), con los siguientes requisitos:

\begin{itemize}
	\itemsep1pt \parskip1pt \parsep1pt
	\item Ubuntu Saucy (13.10) o Ubuntu Trusty (14.04 LTS)
	\item C++03
	\item Boost 1.53
	\item Lisp SBCL 1.0.x
	\item Python 2.7
	\item CMake 2.8.11 \cite{rosrequirements}
\end{itemize}

Durante el proceso de instalación, se lleva a cabo la satisfacción de dependencias.

\section{Procedimiento de Instalación}

\subsection{Descripción de Entornos de Desarrollo}

Los entornos y dispositivos utilizados para llevar a cabo el proyecto, para pruebas de instalación y/o funcionamiento, son los siguientes:

\begin{itemize}
	\item Máquina Virtual: Oracle \textregistered{} VirtualBox VM\textsuperscript{TM}, 2 GB RAM, 30 GB disco duro, con Ubuntu Trusty Tahr 14.04.2 de 64 bits.
	\item Computador Portatil: Lenovo \textregistered{} Z50-70, procesador Intel \textregistered{}  Core\textsuperscript{TM} i7-4510U, 6 GB RAM, 500 GB disco duro, con Ubuntu Trusty Tahr 14.04.2 de 64 bits.
	\item Microsoft \textregistered{}  XBOX 360 \textregistered{}  Kinect\textsuperscript{TM}
\end{itemize}

En la máquina virtual se llevó a cabo la comprobación del procedimiento de instalación, ya que es un entorno que permite restaurar a un punto anterior con facilidad, en caso de inconvenientes tales como paquetes mal instalados, etc.

\subsection{Instalación}

El proceso de instalación de ROS está excelentemente documentado en su \textit{wiki} oficial accesible desde \url{http://wiki.ros.org/ROS/Installation}, por lo cual seguimos los pasos tomando nota de cualquier dependencia faltante o error.

Para comenzar, en el enlace previamente mencionado, hacemos clic en ``Indigo installation instructions'', puesto que es la versión de soporte extendido y es la compatible con la versión instalada de Ubuntu.

Una vez allí, bajo el apartado ``Supported'', hacemos clic en ``Ubuntu''.

En adelante, solamente seguimos los siguientes pasos haciendo énfasis en la versión instalada de Ubuntu, tomados directamente del sitio:

\begin{enumerate}
\renewcommand{\labelenumii}{\theenumii}
\renewcommand{\theenumii}{\theenumi.\arabic{enumii}.}
	\item Configure sus repositorios de Ubuntu

	Configure sus repositorios de Ubuntu para activar los repositorios ``restricted'', ``universe'' y ``multiverse''. La guía de configuración de repositorios de Ubuntu, disponible en el siguiente enlace \url{https://help.ubuntu.com/community/Repositories/Ubuntu} (en inglés) detalla adecuadamente los pasos necesarios.
	\item Configure el archivo sources.list

	Configure su computador para aceptar software desde packages.ros.org. Esta distribución de ROS \textbf{sólo} soporta Saucy (Ubuntu 13.10) y Trusty (Ubuntu 14.04) para paquetes Debian.

	\begin{blackcodebox}
	\begin{lstlisting}[language=bash]
sudo sh -c 'echo "deb http://packages.ros.org/ros/ubuntu $(lsb_release -sc) main" > /etc/apt/sources.list.d/ros-latest.list'
	\end{lstlisting}
	\end{blackcodebox}

	\item Configure sus llaves

	\begin{blackcodebox}
	\begin{lstlisting}[language=bash]
sudo apt-key adv --keyserver hkp://pool.sks-keyservers.net --recv-key\\ 0xB01FA116
	\end{lstlisting}
	\end{blackcodebox}

	\item Instalación
	Primero debe asegurarse que su índice de paquetes Debian esté actualizado:

	\begin{blackcodebox}
	\begin{lstlisting}[language=bash]
sudo apt-get update
	\end{lstlisting}
	\end{blackcodebox}

	Si está utilizando Ubuntu Trusty 14.04.2 y experimenta problemas con las dependencias durante la instalación de ROS, quizás deba instalar dependencias adicionales del sistema.

	\begin{redwarningbox}
	No instale estos paquetes si está utilizando 14.04, ya que destruirá su servidor X (gráfico):
	\end{redwarningbox}
	\begin{blackcodebox}
	\begin{lstlisting}[language=bash]
sudo apt-get install xserver-xorg-dev-lts-utopic mesa-common-dev-lts-utopic libxatracker-dev-lts-utopic libopenvg1-mesa-dev-lts-utopic libgles2-mesa-dev-lts-utopic libgles1-mesa-dev-lts-utopic libgl1-mesa-dev-lts-utopic libgbm-dev-lts-utopic libegl1-mesa-dev-lts-utopic
	\end{lstlisting}
	\end{blackcodebox}
	\begin{redwarningbox}
	No instale los paquetes anteriores si está utilizando 14.04, ya que destruirá su servidor X (gráfico). Alternativamente, intente instalar sólo lo siguiente para corregir problemas de dependencias:
	\end{redwarningbox}
	\begin{blackcodebox}
	\begin{lstlisting}[language=bash]
sudo apt-get install libgl1-mesa-dev-lts-utopic
	\end{lstlisting}
	\end{blackcodebox}

	Existen muchas bibliotecas y herramientas en ROS. Se han provisto cuatro configuraciones por defecto para iniciar. También se pueden instalar paquetes de ROS de forma individual.

	\begin{itemize}
		\item Instalación de Escritorio Completa: (Recomendada): ROS, rqt, rviz, bibliotecas genéricas para robots, simuladores 2D/3D, navegación y percepción 2D/3D

		Indigo usa Gazebo 2, la cual es la versión por defecto en Trusty y es la recomendada. Si desea actualizar a Gazebo 3 vea las instrucciones en \url{http://wiki.gazebosim.org/wiki/Install/Gazebo_and_ROS#Gazebo_3.x_series} acerca de cómo actualizar el simulador.


		\begin{blackcodebox}
		\begin{lstlisting}[language=bash]
sudo apt-get install ros-indigo-desktop-full
		\end{lstlisting}
		\end{blackcodebox}

		\item Instalación de Escritorio: ROS, rqt, rviz, y bibliotecas genéricas para robots.

		\begin{blackcodebox}
		\begin{lstlisting}[language=bash]
sudo apt-get install ros-indigo-desktop
		\end{lstlisting}
		\end{blackcodebox}

		\item ROS-Base: (Esencial) Las bibliotecas de paquete, generación y comunicación. No incluye herramientas de entorno gráfico.

		\begin{blackcodebox}
		\begin{lstlisting}[language=bash]
sudo apt-get install ros-indigo-ros-base
		\end{lstlisting}
		\end{blackcodebox}

		\item Paquete Individual: También puede instalar un paquete específico de ROS package (reemplace subguiones con guiones del nombre del paquete):

		\begin{blackcodebox}
		\begin{lstlisting}[language=bash]
sudo apt-get install ros-indigo-PAQUETE
		\end{lstlisting}
		\end{blackcodebox}
		por ejemplo:

		\begin{blackcodebox}
		\begin{lstlisting}[language=bash]
sudo apt-get install ros-indigo-slam-gmapping
		\end{lstlisting}
		\end{blackcodebox}

		Para listar paquetes disponibles, use:

		\begin{blackcodebox}
		\begin{lstlisting}[language=bash]
apt-cache search ros-indigo
		\end{lstlisting}
		\end{blackcodebox}
	\end{itemize}
	\item Inicialice rosdep

	Antes de poder utilizar ROS, se debe inicializar rosdep. rosdep permite instalar dependencias del sistema con facilidad para código fuente que desee compilar y es requerido para poder ejecutar algunos componentes centrales en ROS.

	\begin{blackcodebox}
	\begin{lstlisting}[language=bash]
sudo rosdep init
rosdep update
	\end{lstlisting}
	\end{blackcodebox}

	\item Configuración de entorno

	Es conveniente si las variables de entorno de ROS son agregadas automáticamente a su sesión Bash cada vez que se invoca un nuevo terminal:

	\begin{blackcodebox}
	\begin{lstlisting}[language=bash]
echo "source /opt/ros/indigo/setup.bash" >> ~/.bashrc
source ~/.bashrc
	\end{lstlisting}
	\end{blackcodebox}

	Si tiene más de una distribución de ROS instalada, \url{~/.bashrc} sólo debe tomar como fuente el \url{setup.bash} para la versión que esté en uso actualmente.

	Si simplemente desea cambiar el entorno del terminal actual, puede escribir:

	\begin{blackcodebox}
	\begin{lstlisting}[language=bash]
source /opt/ros/indigo/setup.bash
	\end{lstlisting}
	\end{blackcodebox}

	\item Obtener rosinstall

rosinstall es una herramienta de línea de comandos frecuentemente utilizada en ROS que es distribuida por separado. Le permite descargar con facilidad muchos árboles fuentes para paquetes ROS con un solo comando.

Para instalar esta herramienta en Ubuntu, ejecute:

	\begin{blackcodebox}
	\begin{lstlisting}[language=bash]
sudo apt-get install python-rosinstall
	\end{lstlisting}
	\end{blackcodebox}
\end{enumerate}

\section{Módulos Disponibles para SLAM en ROS}

Debido a la naturaleza propia del software, es improbable, para no decir imposible, poder realizar un listado exhaustivo de todos los módulos disponibles. Sin embargo, se listan acá los más populares, con el fin de brindar alternativas para elaborar mapas de entorno con distintos tipos de sensores.

Es de notar que algunos paquetes requieren obtener datos de un sensor láser, por lo cual el Kinect no cumpliría con ese requerimiento; sin embargo, hay un paquete o nodo de ROS que permite convertir las imágenes con datos de profundidad capturadas por un sensor RGB-D (tal como el Kinect) y emular un sensor láser. Este puede encontrarse acá: \url{http://wiki.ros.org/depthimage_to_laserscan}

\begin{itemize}
	\itemsep1pt \parskip1pt \parsep1pt
	\item gmapping (\url{http://wiki.ros.org/gmapping})
	\item hector\_slam (\url{http://wiki.ros.org/hector_slam})
	\item RTAB-Map (\url{http://wiki.ros.org/rtabmap})
	\item rgbdslam (\url{http://wiki.ros.org/rgbdslam})
	\item rgbdslam V2 (versión actualizada) (\url{http://felixendres.github.io/rgbdslam_v2/})
\end{itemize}

Tras evaluar documentación disponible, ejemplos de código y funcionamiento y versiones soportadas de cada módulo en ROS, se decidió utilizar RTAB-Map (descrito en el siguiente capítulo).

Es de destacar, que RTAB-Map cuenta con soporte directo por parte del desarrollador del mismo, por lo cual será utilizado en este proyecto para la generación de mapas.