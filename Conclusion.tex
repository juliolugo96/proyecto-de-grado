\chapter{Conclusión y Recomendaciones}

\section{Conclusión}

Las Redes Sociales como plataformas tecnológicas y la necesidad de la creación de valor han dilucidado, como se ha podido contemplar durante el desarrollo de este proyecto, ser un potencial de actividad económica gigantesco, que puede impulsar cambios masivos en la calidad de vida de las personas, al emparejar emprendedores con inversores, así como empleados con empresas propias.

Asimismo, el desarrollo de plataformas haciendo uso de marcos de trabajo de desarrollo web de este nivel de complejidad exigen de un entendimiento muy claro de los requerimientos del producto, razón por la cual una gran parte del tiempo de desarrollo fue usado para este fin, a modo de detallar de la manera máx explícita posible cada uno de los requerimientos necesitados.

\section{Recomendaciones}

El software descrito en este documento posee la capacidad de fungir como base de múltiples desarrollos a nivel de software que permitiesen facilitar la creación de redes sociales. Para ello, se pueden seguir las siguientes recomendaciones:

\begin{itemize}
	\itemsep1pt \parskip1pt \parsep1pt
	\item Continuar con el desarrollo del proyecto en las siguientes funcionalidades: desarrollo de un sistema de Autenticación a dos pasos, optimización de tiempos de carga, incorporación de un chat entre usuarios, desarrollo de preferencias de usuario y soporte a integración con otras plataformas.

	\item Publicar la base de código como un paquete de acceso público o privado de NPM.

	\item Proponer un proyecto de marketing digital para hacer crecer la presencia de la plataforma Ignis Gravitas en la red.
\end{itemize}