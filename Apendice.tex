\chapter{LaSDAI}

\section{Descripcion de LaSDAI} \label{App:DescripcionLasdai}

El Laboratorio de Sistemas Discretos, Automatizacion e Integracion (LaSDAI), se funda en el año 1993 bajo la direccion del Dr.\ Edgar Chacon cno el apoyo del Programa de Nuevas Tecnologías del CONICIT (Consejo Nacional de Investigaciones Científicas y Tecnológicas), bajo el proyecto N\textordmasculine\ I-22. LaSDAI está adscrito a la Escuela de Ingeniería de Sistemas de la Facultad de Ingeniería de la Universidad de Los Andes. A partir del año 2007 actualiza sus líneas de investigacion y es re-estructurado. Actualmente está conformado por personal docente, estudiantes y colaboradores de la Universidad de Los Andes.\cite{lasdaiInicio}

\subsection{Personal}

\subsubsection{Miembros}

\begin{itemize}
    \itemsep1pt \parskip0pt \parsep0pt
    \item Dr.\ Eladio Dapena G. (Coordinador)
    \item Dr.\ Rafael Rivas Estrada
    \item Ing.\ Jose G. Gonzalez
    \item Dr.\ Edgar Chacón
\end{itemize}

\subsubsection{Colaboradores}

\begin{itemize}
    \itemsep1pt \parskip0pt \parsep0pt
    \item PhD.\ Addison Rios
    \item Dr.\ José Aguilar
    \item Dr.\ José María Armingol (UC3M)
    \item Dr.\ Arturo de La Escalera (UC3M)
    \item Dra.\ Ana Corrales (UC3M)
    \item Ing.\ David Godoy
    \item Lic.\ Nadia González
    \item MSc.\ Asdrúbal Fernández
\end{itemize}

\subsection{Mision}

El Laboratorio de Sistemas Discretos, Automatizacion e Integracion, LaSDAI, es un espacio para la docencia, la investigacion y el desarrollo de productos, en las áreas de robótica, automatizacion industrial y vision por computador, con el objeto de coadyuvar en el desarrollo tecnológico del país.
LaSDAI, tiene como meta difundir sus resultados  y vincularse con el sector productivo nacional, con la consiga de I+D+I Investigacion, Desarrollo e Innovacion. Sus actividades de soporte a la docencia tanto en pregrado como postgrado, junto con el desarrollo de proyectos de investigacion y las labores de extension, constituyen un complemento que integra diferentes aristas para el desarrollo. Las actividades de extension, mediante asesorías y cursos, colaboran a lograr el autofinanciamiento como estrategia de consolidacion de nuestras actividades.

\subsection{Vision}

LaSDAI será un referente nacional, como un Instituto de Docencia, Investigacion y Desarrollo, en las áreas de robótica y automatizacion, de carácter autonomo y autofinanciado.

\subsection{Objetivos}

\subsubsection{Objetivo General}

El Laboratorio de Sistemas Discretos, Automatizacion e Integracion fue creado con el fin de desarrollar conceptos y herramientas que soporten la construccion de sistemas automáticos en ambientes de produccion continua, tal como son los de la industria de procesos, mediante una generalizacion de las técnicas de los ambientes de manufactura.

La automatizacion en la industria de Procesos Continuos es un proceso complejo, por la cantidad de elementos involucrados estén un proceso de mejoramiento, donde la automatizacion juega un papel esencial. Los paradigmas para la implantacion de una automatizacion integral en una industria compleja no han sido totalmente definidos, pues el mayor énfasis en la comunidad científica a nivel internacional ha sido dado a la industria de manufactura; de aquí la importancia que tiene para el grupo el desarrollo de trabajos de investigacion en el área, así como la formacion de personal.

\subsubsection{Objetivos Específicos}

\begin{itemize}
    \itemsep1pt \parskip0pt \parsep0pt
    \item Realizar actividades de Investigacion, Desarrollo e Innovacion (I+D+I), en las áreas de Robótica, Automatizacion Industrial y Vision por Computador.
    \item Realizar asesorías a la Industria nacional e internacional.
    \item Divulgacion del conocimiento mediante la docencia, organizacion de eventos, publicacion de resultados, etc.
    \item Formacion de personal en las áreas a fines a nivel de pregrado, postgrado, doctorado en diversas universidades.
    \item Desarrollar proyectos en cooperacion con otros centros y laboratorios de investigacion.
    \item Formacion de personal mediante cursos de extension en empresas.
    \item Establecer relaciones de cooperacion con el sector productivo nacional mediante el desarrollo de proyectos.
    \item Participar en eventos científicos nacionales e internacionales.
\end{itemize}